\section{Proofs}\label{A:Proofs}
\begin{proof}[\textbf{Theorem 1}]
The proof of Theorem~\ref{T:BiasWEME} follows directly from observing that $\est{\WE}$ is always smaller than $\est{\ME}$. In fact, the \ME~estimator can be seen as a weighted estimator that gives probability one to the variable associated to the largest sample mean $\hat\mu_i$, so that any other weighting cannot produce a larger value. 
\end{proof}

\begin{proof}[\textbf{Theorem 2}]
  If we compare the expected value of \CV~reported in Equation~\eqref{E:biasCV} with the value of the  estimator \WE~in Equation~\eqref{E:WE}, we can notice strong similarities.
  The main difference is that in \CV~the sample mean of variable $X_i$ and its probability of being the maximum are computed w.r.t.~two independent set of samples, while in \WE~these two quantities are positively correlated. It follows that \WE~has a positive bias w.r.t. \CV.
\end{proof}

\begin{proof}[\textbf{Theorem 3}]
 Starting from the definition of \WE~\eqref{E:WE}, we can derive the bound to the variance as follows
 \begin{align*}
  \mathrm{Var}\left(\est{\WE}\right) &= \mathrm{Var}\left(\sum_{i=1}^M \hat\mu_i(S) w_i^S\right)\\
  &\leq \mathrm{Var}\left(\sum_{i=1}^M \hat\mu_i(S)\right)\\
  &= \sum_{i=1}^M \mathrm{Var}\left(\hat\mu_i(S)\right),
 \end{align*}
 where the inequality is a consequence of the maximization of each weight $w_i^S$ with one and the last equality comes from the independence of the sample means.
\end{proof}

\begin{proof}[\textbf{Theorem 4}]
 Since the weights $w_i$ computed by \OWE~are not random variables, it follows
 \begin{align*}
  \mathrm{Var}\left(\est{\OWE}\right) &= \mathrm{Var}\left(\sum_{i=1}^M \hat\mu_i(S) w_i\right)\\
  &= \sum_{i=1}^M w_i^2\mathrm{Var}\left(\hat\mu_i(S)\right)\\
  &\leq \max_{i\in{1,\dots,M}} \frac{\sigma^2_i}{|S_i|},
 \end{align*}
 where the inequality is motivated by $w_i^2\leq 1,\;\forall i.$
\end{proof}


\section{Forex}\label{A:Forex}
The indicators chosen to define the states are: Moving Average Convergence/Divergence indicator, Relative Strength Index, Momentum, Channel Commodity Index, Stochastic Oscillator, Bollinger Bands, Moving Average Cross-Over
The actions suggested by the indicators are computed by setting the parameters and the entry-exit conditions following the most used and common rules for these indicators.
In particular all the signals are defined using implemented Matlab Financial Toolbox functions and setting parameters and conditions like below.

\begin{figure}[t]
    \begin{minipage}{\columnwidth}
    \centering 
    \setlength\figureheight{6cm}
    \setlength\figurewidth{8cm}
    % This file was created by matlab2tikz.
%
%The latest updates can be retrieved from
%  http://www.mathworks.com/matlabcentral/fileexchange/22022-matlab2tikz-matlab2tikz
%where you can also make suggestions and rate matlab2tikz.
%
\begin{tikzpicture}

\begin{axis}[%
width=0.83\figurewidth,
height=\figureheight,
at={(0\figurewidth,0\figureheight)},
scale only axis,
xmin=0,
xmax=1400,
xlabel={Days},
ymin=-0.1,
ymax=0.25,
ytick={-0.1,-0.05,-1.38777878078145e-17,0.05,0.1,0.15,0.2,0.25},
yticklabels={{         -10\%},{          -5\%},{0\%},{           5\%},{          10\%},{          15\%},{          20\%},{          25\%}},
ylabel={Cumulative Profit},
axis background/.style={fill=white},
axis x line*=bottom,
axis y line*=left,
legend style={at={(0.97,0.03)},anchor=south east,font=\tiny,legend cell align=left,align=left,draw=white!15!black}
]
\addplot [color=blue,solid]
  table[row sep=crcr]{%
1	6.92479999999998e-310\\
2	1.38496e-309\\
3	-4.6e-05\\
4	-0.001148\\
5	-0.00116\\
6	-0.000554\\
7	-0.000588\\
8	-0.000618\\
9	0.002302\\
10	0.001886\\
11	0.000456\\
12	0.000414000000000002\\
13	-0.001142\\
14	-0.00117\\
15	-0.001204\\
16	6.60000000000001e-05\\
17	-0.00933799999999999\\
18	-0.00189799999999999\\
19	0.00493300000000003\\
20	0.00292100000000002\\
21	0.00289900000000002\\
22	0.000947000000000017\\
23	-0.000174999999999982\\
24	-0.000666999999999982\\
25	0.000111000000000018\\
26	-0.00124499999999998\\
27	-0.00459299999999998\\
28	-0.00462499999999998\\
29	-0.00585099999999998\\
30	-0.011875\\
31	-0.00819099999999998\\
32	-0.013013\\
33	-0.020515\\
34	-0.021103\\
35	-0.021127\\
36	-0.021911\\
37	-0.037037\\
38	-0.035265\\
39	-0.037993\\
40	-0.033445\\
41	-0.033598\\
42	-0.033604\\
43	-0.034098\\
44	-0.032818\\
45	-0.025864\\
46	-0.02827\\
47	-0.0277\\
48	-0.021606\\
49	-0.021612\\
50	-0.021914\\
51	-0.021994\\
52	-0.022308\\
53	-0.026902\\
54	-0.031424\\
55	-0.022888\\
56	-0.022896\\
57	-0.022324\\
58	-0.02715\\
59	-0.019994\\
60	-0.024142\\
61	-0.023547\\
62	-0.022766\\
63	-0.022766\\
64	-0.022396\\
65	-0.026846\\
66	-0.026894\\
67	-0.025343\\
68	-0.025565\\
69	-0.026043\\
70	-0.026047\\
71	-0.023651\\
72	-0.014399\\
73	-0.004601\\
74	0.00973599999999997\\
75	0.01179\\
76	0.00963399999999996\\
77	0.00963199999999996\\
78	0.011113\\
79	0.013433\\
80	0.012585\\
81	0.011163\\
82	0.010019\\
83	0.010951\\
84	0.010943\\
85	0.011695\\
86	0.011787\\
87	0.012847\\
88	0.012748\\
89	0.013312\\
90	0.023796\\
91	0.023764\\
92	0.021208\\
93	0.027928\\
94	0.026005\\
95	0.029189\\
96	0.018419\\
97	0.02769\\
98	0.027664\\
99	0.028229\\
100	0.033935\\
101	0.04551\\
102	0.047478\\
103	0.038964\\
104	0.043672\\
105	0.043658\\
106	0.043944\\
107	0.040947\\
108	0.035681\\
109	0.035282\\
110	0.037942\\
111	0.038674\\
112	0.038658\\
113	0.035302\\
114	0.025562\\
115	0.025494\\
116	0.026829\\
117	0.026787\\
118	0.025266\\
119	0.02526\\
120	0.02498\\
121	0.023892\\
122	0.021669\\
123	0.021894\\
124	0.03007\\
125	0.03948\\
126	0.039452\\
127	0.04151\\
128	0.049763\\
129	0.052888\\
130	0.047352\\
131	0.05121\\
132	0.056123\\
133	0.056109\\
134	0.059078\\
135	0.05528\\
136	0.058087\\
137	0.052177\\
138	0.061626\\
139	0.058444\\
140	0.0584\\
141	0.058482\\
142	0.058238\\
143	0.058436\\
144	0.059428\\
145	0.059496\\
146	0.059496\\
147	0.05949\\
148	0.059434\\
149	0.059332\\
150	0.059402\\
151	0.04941\\
152	0.047147\\
153	0.045943\\
154	0.045941\\
155	0.045819\\
156	0.043663\\
157	0.042989\\
158	0.043884\\
159	0.043199\\
160	0.052308\\
161	0.052302\\
162	0.047491\\
163	0.043082\\
164	0.044143\\
165	0.037288\\
166	0.036217\\
167	0.045217\\
168	0.045185\\
169	0.043685\\
170	0.047265\\
171	0.036227\\
172	0.034403\\
173	0.034531\\
174	0.038737\\
175	0.038727\\
176	0.039495\\
177	0.039577\\
178	0.039549\\
179	0.042848\\
180	0.051589\\
181	0.055405\\
182	0.055377\\
183	0.055626\\
184	0.041852\\
185	0.024641\\
186	0.024189\\
187	0.024092\\
188	0.024364\\
189	0.02434\\
190	0.02432\\
191	0.024448\\
192	0.023962\\
193	0.023911\\
194	0.024009\\
195	0.00959899999999996\\
196	0.00956899999999996\\
197	0.00972899999999996\\
198	0.010486\\
199	0.00877599999999996\\
200	0.010194\\
201	0.013188\\
202	0.0159\\
203	0.015882\\
204	0.01345\\
205	0.025353\\
206	0.028067\\
207	0.035167\\
208	0.036941\\
209	0.034693\\
210	0.034621\\
211	0.036365\\
212	0.041892\\
213	0.041222\\
214	0.037214\\
215	0.029682\\
216	0.033338\\
217	0.033222\\
218	0.032845\\
219	0.034811\\
220	0.035181\\
221	0.038057\\
222	0.041353\\
223	0.040587\\
224	0.040489\\
225	0.040447\\
226	0.039695\\
227	0.039711\\
228	0.039953\\
229	0.039664\\
230	0.038938\\
231	0.038876\\
232	0.035695\\
233	0.039729\\
234	0.044303\\
235	0.029417\\
236	0.028729\\
237	0.037735\\
238	0.037703\\
239	0.034337\\
240	0.032923\\
241	0.032979\\
242	0.035445\\
243	0.035743\\
244	0.034223\\
245	0.034181\\
246	0.032203\\
247	0.042366\\
248	0.042062\\
249	0.042562\\
250	0.047572\\
251	0.046586\\
252	0.04652\\
253	0.04973\\
254	0.04928\\
255	0.048572\\
256	0.046658\\
257	0.04718\\
258	0.060204\\
259	0.060148\\
260	0.062249\\
261	0.057787\\
262	0.066877\\
263	0.065352\\
264	0.073082\\
265	0.066929\\
266	0.066899\\
267	0.069824\\
268	0.075394\\
269	0.072089\\
270	0.064821\\
271	0.064871\\
272	0.050705\\
273	0.050697\\
274	0.050487\\
275	0.049219\\
276	0.046205\\
277	0.038151\\
278	0.038076\\
279	0.038069\\
280	0.038055\\
281	0.033719\\
282	0.022255\\
283	0.021401\\
284	0.025772\\
285	0.028288\\
286	0.021226\\
287	0.021206\\
288	0.018918\\
289	0.019397\\
290	0.033726\\
291	0.026542\\
292	0.033563\\
293	0.024342\\
294	0.024326\\
295	0.024806\\
296	0.018567\\
297	0.015137\\
298	0.026385\\
299	0.025432\\
300	0.00524199999999999\\
301	0.00520999999999999\\
302	0.00491399999999999\\
303	0.00436399999999999\\
304	0.00438199999999999\\
305	0.00438599999999999\\
306	0.00495199999999999\\
307	0.00495799999999999\\
308	0.00493599999999999\\
309	0.00492499999999999\\
310	0.00545099999999999\\
311	0.00643099999999999\\
312	0.020193\\
313	0.019453\\
314	0.020913\\
315	0.020909\\
316	0.021403\\
317	0.022515\\
318	0.018373\\
319	0.017718\\
320	0.0173\\
321	0.024457\\
322	0.024409\\
323	0.024367\\
324	0.022711\\
325	0.023493\\
326	0.02239\\
327	0.024878\\
328	0.024936\\
329	0.024932\\
330	0.024388\\
331	0.0334\\
332	0.03414\\
333	0.027808\\
334	0.043604\\
335	0.046539\\
336	0.046537\\
337	0.043359\\
338	0.025659\\
339	0.022263\\
340	0.021521\\
341	0.023471\\
342	0.027622\\
343	0.027582\\
344	0.027772\\
345	0.021938\\
346	0.025248\\
347	0.030623\\
348	0.032496\\
349	0.034156\\
350	0.034092\\
351	0.033914\\
352	0.03468\\
353	0.035212\\
354	0.034982\\
355	0.033923\\
356	0.034157\\
357	0.034143\\
358	0.034009\\
359	0.037433\\
360	0.014125\\
361	0.016735\\
362	0.015127\\
363	0.029714\\
364	0.029696\\
365	0.027729\\
366	0.023571\\
367	0.023437\\
368	0.024734\\
369	0.029088\\
370	0.021414\\
371	0.021282\\
372	0.021588\\
373	0.027725\\
374	0.027406\\
375	0.029824\\
376	0.031633\\
377	0.026753\\
378	0.026681\\
379	0.025975\\
380	0.028465\\
381	0.025641\\
382	0.026001\\
383	0.024179\\
384	0.017123\\
385	0.017089\\
386	0.015433\\
387	0.016345\\
388	0.005313\\
389	0.006828\\
390	0.005314\\
391	0.011532\\
392	0.011524\\
393	0.011596\\
394	0.026302\\
395	0.02649\\
396	0.029654\\
397	0.026191\\
398	0.026159\\
399	0.026143\\
400	0.026167\\
401	0.026399\\
402	0.032527\\
403	0.028817\\
404	0.029403\\
405	0.029267\\
406	0.029245\\
407	0.029181\\
408	0.02837\\
409	0.026888\\
410	0.025494\\
411	0.023221\\
412	0.017979\\
413	0.017955\\
414	0.01231\\
415	0.012156\\
416	0.025204\\
417	0.021244\\
418	0.024236\\
419	0.026536\\
420	0.026504\\
421	0.029904\\
422	0.021976\\
423	0.00938300000000001\\
424	0.00940300000000001\\
425	0.00938900000000001\\
426	0.00972700000000001\\
427	0.00970100000000001\\
428	0.00955300000000001\\
429	0.00862900000000001\\
430	0.00728100000000001\\
431	0.00720700000000001\\
432	0.00647100000000001\\
433	0.00636900000000001\\
434	0.00633500000000001\\
435	0.00596900000000001\\
436	0.00624900000000001\\
437	0.000779000000000013\\
438	-0.00159899999999999\\
439	-0.012432\\
440	-0.006854\\
441	-0.006866\\
442	-0.006484\\
443	-0.003096\\
444	-0.00604\\
445	-0.00599\\
446	-0.00457\\
447	-0.004794\\
448	-0.004814\\
449	-0.004864\\
450	-0.004392\\
451	-0.004266\\
452	-0.006578\\
453	-0.007294\\
454	-0.016346\\
455	-0.016354\\
456	-0.016066\\
457	-0.016114\\
458	-0.00987399999999999\\
459	-0.00998399999999999\\
460	-0.008364\\
461	-0.008994\\
462	-0.009012\\
463	-0.004998\\
464	0.00930599999999997\\
465	0.020934\\
466	0.026008\\
467	0.014668\\
468	0.017278\\
469	0.017264\\
470	0.020877\\
471	0.019691\\
472	0.021447\\
473	0.019489\\
474	0.029067\\
475	0.046322\\
476	0.046312\\
477	0.044825\\
478	0.044269\\
479	0.057168\\
480	0.060932\\
481	0.071309\\
482	0.078223\\
483	0.078221\\
484	0.081175\\
485	0.076707\\
486	0.083911\\
487	0.077263\\
488	0.101275\\
489	0.109714\\
490	0.109712\\
491	0.113968\\
492	0.0936099999999999\\
493	0.100898\\
494	0.0974799999999999\\
495	0.104652\\
496	0.0935859999999999\\
497	0.0935799999999999\\
498	0.0931299999999999\\
499	0.0961049999999999\\
500	0.0940629999999999\\
501	0.0906299999999999\\
502	0.0865199999999999\\
503	0.0996729999999999\\
504	0.0996709999999999\\
505	0.0992239999999999\\
506	0.0996969999999999\\
507	0.0803799999999999\\
508	0.0900679999999999\\
509	0.0937859999999999\\
510	0.0796169999999999\\
511	0.0796149999999999\\
512	0.0786599999999999\\
513	0.0898429999999999\\
514	0.0992429999999999\\
515	0.102363\\
516	0.114661\\
517	0.117245\\
518	0.117243\\
519	0.119259\\
520	0.117069\\
521	0.118791\\
522	0.11481\\
523	0.112492\\
524	0.112857\\
525	0.112843\\
526	0.112183\\
527	0.115913\\
528	0.114023\\
529	0.113475\\
530	0.108423\\
531	0.109729\\
532	0.109689\\
533	0.109529\\
534	0.109299\\
535	0.109371\\
536	0.109557\\
537	0.109448\\
538	0.108227\\
539	0.108205\\
540	0.107218\\
541	0.10329\\
542	0.10341\\
543	0.098226\\
544	0.099176\\
545	0.106867\\
546	0.106855\\
547	0.106509\\
548	0.102675\\
549	0.103982\\
550	0.098433\\
551	0.104044\\
552	0.101903\\
553	0.101883\\
554	0.101517\\
555	0.11598\\
556	0.111027\\
557	0.09631\\
558	0.094092\\
559	0.096081\\
560	0.096075\\
561	0.096559\\
562	0.092529\\
563	0.089313\\
564	0.095583\\
565	0.096893\\
566	0.100027\\
567	0.100021\\
568	0.100705\\
569	0.102297\\
570	0.106705\\
571	0.107122\\
572	0.11153\\
573	0.121294\\
574	0.121292\\
575	0.121759\\
576	0.118628\\
577	0.114396\\
578	0.118923\\
579	0.108912\\
580	0.106874\\
581	0.10686\\
582	0.107234\\
583	0.105549\\
584	0.107517\\
585	0.109007\\
586	0.104055\\
587	0.107114\\
588	0.107094\\
589	0.106202\\
590	0.099382\\
591	0.102762\\
592	0.0984220000000001\\
593	0.102959\\
594	0.103179\\
595	0.103177\\
596	0.103397\\
597	0.0960790000000001\\
598	0.0945230000000001\\
599	0.0921920000000001\\
600	0.091926\\
601	0.0928440000000001\\
602	0.09284\\
603	0.093696\\
604	0.092782\\
605	0.085692\\
606	0.0809710000000001\\
607	0.08902\\
608	0.096888\\
609	0.096882\\
610	0.095212\\
611	0.095614\\
612	0.09393\\
613	0.092216\\
614	0.089882\\
615	0.092234\\
616	0.09222\\
617	0.092786\\
618	0.09167\\
619	0.091758\\
620	0.09477\\
621	0.104246\\
622	0.101305\\
623	0.101279\\
624	0.102726\\
625	0.102772\\
626	0.09935\\
627	0.096686\\
628	0.088548\\
629	0.090538\\
630	0.09052\\
631	0.090604\\
632	0.078926\\
633	0.078941\\
634	0.078749\\
635	0.085442\\
636	0.0936100000000001\\
637	0.0935880000000001\\
638	0.0935860000000001\\
639	0.0966240000000001\\
640	0.108681\\
641	0.107809\\
642	0.103413\\
643	0.103454\\
644	0.10343\\
645	0.103474\\
646	0.100671\\
647	0.0994570000000001\\
648	0.0998080000000001\\
649	0.103652\\
650	0.108244\\
651	0.108216\\
652	0.10702\\
653	0.0993920000000001\\
654	0.0993780000000001\\
655	0.104306\\
656	0.113518\\
657	0.120108\\
658	0.120066\\
659	0.118934\\
660	0.120906\\
661	0.126925\\
662	0.126197\\
663	0.126803\\
664	0.121423\\
665	0.121411\\
666	0.121785\\
667	0.118769\\
668	0.123182\\
669	0.12268\\
670	0.123786\\
671	0.1262\\
672	0.126196\\
673	0.126627\\
674	0.122109\\
675	0.122283\\
676	0.127696\\
677	0.128636\\
678	0.122734\\
679	0.12271\\
680	0.122416\\
681	0.119984\\
682	0.117828\\
683	0.104916\\
684	0.104234\\
685	0.10826\\
686	0.108248\\
687	0.114618\\
688	0.113282\\
689	0.123054\\
690	0.116066\\
691	0.129876\\
692	0.142868\\
693	0.142862\\
694	0.144916\\
695	0.143556\\
696	0.136966\\
697	0.130637\\
698	0.127644\\
699	0.120304\\
700	0.120294\\
701	0.119816\\
702	0.123402\\
703	0.130948\\
704	0.128764\\
705	0.1272\\
706	0.135716\\
707	0.135698\\
708	0.135502\\
709	0.139667\\
710	0.117516\\
711	0.119698\\
712	0.120976\\
713	0.122614\\
714	0.122612\\
715	0.122126\\
716	0.12221\\
717	0.125237\\
718	0.124115\\
719	0.119492\\
720	0.12077\\
721	0.120768\\
722	0.120963\\
723	0.118747\\
724	0.109737\\
725	0.091277\\
726	0.090563\\
727	0.102663\\
728	0.102651\\
729	0.102378\\
730	0.09313\\
731	0.094054\\
732	0.100745\\
733	0.090196\\
734	0.075824\\
735	0.075764\\
736	0.076005\\
737	0.077621\\
738	0.076318\\
739	0.079443\\
740	0.080886\\
741	0.080668\\
742	0.080654\\
743	0.080743\\
744	0.082073\\
745	0.084507\\
746	0.078019\\
747	0.076989\\
748	0.073699\\
749	0.073673\\
750	0.071333\\
751	0.071226\\
752	0.070938\\
753	0.070716\\
754	0.069882\\
755	0.069998\\
756	0.069992\\
757	0.070044\\
758	0.071165\\
759	0.069463\\
760	0.069459\\
761	0.060287\\
762	0.053635\\
763	0.053605\\
764	0.052179\\
765	0.054593\\
766	0.055738\\
767	0.050374\\
768	0.055422\\
769	0.055999\\
770	0.055993\\
771	0.05502\\
772	0.055774\\
773	0.054088\\
774	0.05507\\
775	0.054041\\
776	0.058389\\
777	0.058361\\
778	0.058844\\
779	0.058368\\
780	0.057573\\
781	0.057833\\
782	0.053081\\
783	0.042511\\
784	0.042499\\
785	0.041315\\
786	0.053623\\
787	0.0445010000000001\\
788	0.0459990000000001\\
789	0.0460650000000001\\
790	0.0452550000000001\\
791	0.0452470000000001\\
792	0.043285\\
793	0.048021\\
794	0.050793\\
795	0.041973\\
796	0.041913\\
797	0.041773\\
798	0.041659\\
799	0.041592\\
800	0.041398\\
801	0.043432\\
802	0.048845\\
803	0.045765\\
804	0.042945\\
805	0.042937\\
806	0.041849\\
807	0.048198\\
808	0.045952\\
809	0.043701\\
810	0.043537\\
811	0.046121\\
812	0.046111\\
813	0.046525\\
814	0.054088\\
815	0.050357\\
816	0.052424\\
817	0.048183\\
818	0.045377\\
819	0.045363\\
820	0.044973\\
821	0.047245\\
822	0.038549\\
823	0.037182\\
824	0.035564\\
825	0.037022\\
826	0.03701\\
827	0.039062\\
828	0.039342\\
829	0.05116\\
830	0.053168\\
831	0.048023\\
832	0.047037\\
833	0.047037\\
834	0.047999\\
835	0.046467\\
836	0.047256\\
837	0.046766\\
838	0.038501\\
839	0.040598\\
840	0.040584\\
841	0.038316\\
842	0.039824\\
843	0.040935\\
844	0.048312\\
845	0.050724\\
846	0.057225\\
847	0.057219\\
848	0.043789\\
849	0.044369\\
850	0.047956\\
851	0.044725\\
852	0.050265\\
853	0.048569\\
854	0.048547\\
855	0.045885\\
856	0.049178\\
857	0.043761\\
858	0.050199\\
859	0.053142\\
860	0.05345\\
861	0.053432\\
862	0.053276\\
863	0.054161\\
864	0.053131\\
865	0.049548\\
866	0.048968\\
867	0.04806\\
868	0.048024\\
869	0.048015\\
870	0.047659\\
871	0.046485\\
872	0.042701\\
873	0.045586\\
874	0.047344\\
875	0.047326\\
876	0.047874\\
877	0.055388\\
878	0.056346\\
879	0.056062\\
880	0.046948\\
881	0.043194\\
882	0.04317\\
883	0.045846\\
884	0.04351\\
885	0.039236\\
886	0.041\\
887	0.043814\\
888	0.035778\\
889	0.035754\\
890	0.035542\\
891	0.035269\\
892	0.036459\\
893	0.037375\\
894	0.037229\\
895	0.039567\\
896	0.039561\\
897	0.039949\\
898	0.039536\\
899	0.0396\\
900	0.040072\\
901	0.041187\\
902	0.045269\\
903	0.045257\\
904	0.044829\\
905	0.050977\\
906	0.050479\\
907	0.050537\\
908	0.047077\\
909	0.046277\\
910	0.046261\\
911	0.047161\\
912	0.050295\\
913	0.059443\\
914	0.063898\\
915	0.061225\\
916	0.066223\\
917	0.066223\\
918	0.066029\\
919	0.068163\\
920	0.079139\\
921	0.083355\\
922	0.078355\\
923	0.084925\\
924	0.084923\\
925	0.084996\\
926	0.088066\\
927	0.08983\\
928	0.083102\\
929	0.078905\\
930	0.064113\\
931	0.064079\\
932	0.066929\\
933	0.058282\\
934	0.059984\\
935	0.058981\\
936	0.054205\\
937	0.051928\\
938	0.051908\\
939	0.04954\\
940	0.0565720000000001\\
941	0.0643960000000001\\
942	0.0541260000000001\\
943	0.0670460000000001\\
944	0.0655500000000001\\
945	0.0655280000000001\\
946	0.0664460000000001\\
947	0.0750680000000001\\
948	0.0909060000000001\\
949	0.0979150000000001\\
950	0.101613\\
951	0.103215\\
952	0.103181\\
953	0.104072\\
954	0.102566\\
955	0.102295\\
956	0.0988330000000002\\
957	0.0976590000000002\\
958	0.0958270000000002\\
959	0.0958030000000002\\
960	0.0967820000000002\\
961	0.0897940000000002\\
962	0.0818500000000002\\
963	0.0959440000000001\\
964	0.0961390000000001\\
965	0.0973530000000001\\
966	0.0973430000000002\\
967	0.0959370000000001\\
968	0.0973350000000001\\
969	0.0960690000000001\\
970	0.0990670000000001\\
971	0.101057\\
972	0.104765\\
973	0.104757\\
974	0.105385\\
975	0.104201\\
976	0.109149\\
977	0.103687\\
978	0.0996330000000002\\
979	0.0938140000000001\\
980	0.0937760000000002\\
981	0.0959450000000002\\
982	0.107281\\
983	0.0895110000000002\\
984	0.0980890000000002\\
985	0.0899390000000002\\
986	0.0815650000000002\\
987	0.0815630000000002\\
988	0.0810170000000002\\
989	0.0809310000000002\\
990	0.0803170000000002\\
991	0.0795660000000002\\
992	0.0829560000000002\\
993	0.0907700000000002\\
994	0.0907620000000002\\
995	0.0893220000000002\\
996	0.0886210000000002\\
997	0.0862760000000002\\
998	0.0811710000000002\\
999	0.0841100000000002\\
1000	0.0861030000000002\\
1001	0.0860970000000002\\
1002	0.0859350000000002\\
1003	0.0809550000000002\\
1004	0.0830490000000002\\
1005	0.0831370000000002\\
1006	0.0816880000000002\\
1007	0.0787000000000002\\
1008	0.0786820000000002\\
1009	0.0805760000000002\\
1010	0.0739780000000002\\
1011	0.0779580000000002\\
1012	0.0710840000000002\\
1013	0.0686420000000002\\
1014	0.0657760000000002\\
1015	0.0657480000000002\\
1016	0.0647740000000002\\
1017	0.0537560000000002\\
1018	0.0545260000000002\\
1019	0.0541770000000002\\
1020	0.0549830000000002\\
1021	0.0556220000000002\\
1022	0.0556220000000002\\
1023	0.0558900000000002\\
1024	0.0591420000000002\\
1025	0.0621560000000002\\
1026	0.0733050000000002\\
1027	0.0717890000000002\\
1028	0.0823580000000002\\
1029	0.0823460000000002\\
1030	0.0803460000000002\\
1031	0.0743610000000002\\
1032	0.0797800000000002\\
1033	0.0903020000000002\\
1034	0.0980100000000002\\
1035	0.0844040000000001\\
1036	0.0843820000000002\\
1037	0.0812030000000002\\
1038	0.0579220000000002\\
1039	0.0443400000000002\\
1040	0.0623370000000002\\
1041	0.0695530000000002\\
1042	0.0624840000000002\\
1043	0.0624760000000002\\
1044	0.0614660000000002\\
1045	0.0655860000000002\\
1046	0.0517940000000002\\
1047	0.0541940000000002\\
1048	0.0553420000000002\\
1049	0.0559360000000002\\
1050	0.0559220000000002\\
1051	0.0534980000000002\\
1052	0.0512700000000002\\
1053	0.0527350000000002\\
1054	0.0553910000000002\\
1055	0.0512750000000002\\
1056	0.0491860000000002\\
1057	0.0491840000000002\\
1058	0.0494880000000002\\
1059	0.0502240000000002\\
1060	0.0550720000000002\\
1061	0.0814180000000003\\
1062	0.0801510000000003\\
1063	0.0780610000000003\\
1064	0.0780430000000003\\
1065	0.0770140000000003\\
1066	0.0733200000000003\\
1067	0.0769240000000003\\
1068	0.0739980000000003\\
1069	0.0749780000000003\\
1070	0.0841480000000003\\
1071	0.0841440000000002\\
1072	0.0832660000000002\\
1073	0.0813350000000002\\
1074	0.0826430000000002\\
1075	0.0856220000000002\\
1076	0.0966840000000002\\
1077	0.0969700000000002\\
1078	0.0969320000000002\\
1079	0.0965580000000002\\
1080	0.101373\\
1081	0.101735\\
1082	0.0964330000000002\\
1083	0.0964150000000002\\
1084	0.0958690000000002\\
1085	0.0958690000000002\\
1086	0.0955800000000002\\
1087	0.0971400000000002\\
1088	0.0968870000000002\\
1089	0.0970200000000002\\
1090	0.0973240000000002\\
1091	0.0973190000000002\\
1092	0.0972730000000002\\
1093	0.0973410000000002\\
1094	0.0973660000000002\\
1095	0.0974400000000002\\
1096	0.0972620000000002\\
1097	0.0968730000000002\\
1098	0.0969990000000002\\
1099	0.0969770000000002\\
1100	0.0968530000000002\\
1101	0.0974070000000002\\
1102	0.0970510000000002\\
1103	0.0965790000000002\\
1104	0.0960910000000002\\
1105	0.0960470000000002\\
1106	0.0960150000000002\\
1107	0.0959370000000002\\
1108	0.0963300000000002\\
1109	0.0966140000000002\\
1110	0.0936340000000002\\
1111	0.0977320000000002\\
1112	0.103454\\
1113	0.103442\\
1114	0.106042\\
1115	0.104292\\
1116	0.100853\\
1117	0.103341\\
1118	0.106183\\
1119	0.123276\\
1120	0.123124\\
1121	0.122723\\
1122	0.117389\\
1123	0.124655\\
1124	0.124945\\
1125	0.133153\\
1126	0.132053\\
1127	0.131997\\
1128	0.131613\\
1129	0.130475\\
1130	0.130975\\
1131	0.130568\\
1132	0.129736\\
1133	0.12192\\
1134	0.121914\\
1135	0.121104\\
1136	0.119258\\
1137	0.11862\\
1138	0.118688\\
1139	0.130519\\
1140	0.132039\\
1141	0.132035\\
1142	0.135189\\
1143	0.132735\\
1144	0.151225\\
1145	0.147415\\
1146	0.152675\\
1147	0.151535\\
1148	0.151503\\
1149	0.147703\\
1150	0.147058\\
1151	0.148266\\
1152	0.15421\\
1153	0.171944\\
1154	0.172942\\
1155	0.17289\\
1156	0.170988\\
1157	0.176646\\
1158	0.175085\\
1159	0.163856\\
1160	0.162072\\
1161	0.169144\\
1162	0.168994\\
1163	0.169455\\
1164	0.179566\\
1165	0.179622\\
1166	0.189832\\
1167	0.189288\\
1168	0.190946\\
1169	0.190876\\
1170	0.192106\\
1171	0.188608\\
1172	0.180828\\
1173	0.178769\\
1174	0.178996\\
1175	0.183801\\
1176	0.183759\\
1177	0.184044\\
1178	0.189567\\
1179	0.191213\\
1180	0.186809\\
1181	0.185964\\
1182	0.186367\\
1183	0.186301\\
1184	0.186079\\
1185	0.186499\\
1186	0.19499\\
1187	0.18962\\
1188	0.189765\\
1189	0.191384\\
1190	0.191304\\
1191	0.193194\\
1192	0.191548\\
1193	0.198486\\
1194	0.192552\\
1195	0.183121\\
1196	0.178964\\
1197	0.178902\\
1198	0.17783\\
1199	0.169522\\
1200	0.159467\\
1201	0.168476\\
1202	0.173486\\
1203	0.167993\\
1204	0.167947\\
1205	0.168199\\
1206	0.161916\\
1207	0.174201\\
1208	0.168471\\
1209	0.174075\\
1210	0.176409\\
1211	0.176375\\
1212	0.175243\\
1213	0.175451\\
1214	0.176003\\
1215	0.188807\\
1216	0.185151\\
1217	0.185655\\
1218	0.185631\\
1219	0.185917\\
1220	0.185765\\
1221	0.187107\\
1222	0.186925\\
1223	0.186683\\
1224	0.185661\\
1225	0.185637\\
1226	0.185673\\
1227	0.186452\\
1228	0.184986\\
1229	0.189\\
1230	0.196705\\
1231	0.196109\\
1232	0.196079\\
1233	0.197364\\
1234	0.198014\\
};
\addlegendentry{Q-Learning};

\addplot [color=red,solid]
  table[row sep=crcr]{%
1	6.93819999999999e-310\\
2	1.38764e-309\\
3	-1.2e-05\\
4	-0.0005\\
5	-0.000406\\
6	-1.79999999999999e-05\\
7	-1.99999999999999e-05\\
8	-2.99999999999999e-05\\
9	0.000354\\
10	0.000344\\
11	0.000122\\
12	0.000112\\
13	-0.000158\\
14	-0.000162\\
15	-0.000178\\
16	0.000414000000000001\\
17	-0.00336\\
18	-0.000647999999999998\\
19	0.001743\\
20	0.001013\\
21	0.001005\\
22	0.000273000000000001\\
23	3.00000000000072e-06\\
24	-7.99999999999993e-05\\
25	6.00000000000007e-05\\
26	-0.000166999999999999\\
27	-0.000774999999999999\\
28	-0.000776999999999999\\
29	-0.001041\\
30	-0.002455\\
31	0.00144499999999999\\
32	-0.00230300000000001\\
33	-0.00757000000000001\\
34	-0.00797200000000001\\
35	-0.00798600000000001\\
36	-0.00852000000000001\\
37	-0.019402\\
38	-0.018254\\
39	-0.018804\\
40	-0.017972\\
41	-0.017995\\
42	-0.017995\\
43	-0.018067\\
44	-0.01785\\
45	-0.016982\\
46	-0.017318\\
47	-0.017248\\
48	-0.016394\\
49	-0.016396\\
50	-0.016445\\
51	-0.016455\\
52	-0.016529\\
53	-0.018217\\
54	-0.019258\\
55	-0.017162\\
56	-0.017164\\
57	-0.01708\\
58	-0.018326\\
59	-0.016706\\
60	-0.017202\\
61	-0.01701\\
62	-0.016821\\
63	-0.016825\\
64	-0.016741\\
65	-0.017714\\
66	-0.017748\\
67	-0.016902\\
68	-0.017038\\
69	-0.017294\\
70	-0.017294\\
71	-0.016152\\
72	-0.014404\\
73	-0.00490999999999999\\
74	0.006063\\
75	0.008052\\
76	0.005938\\
77	0.005906\\
78	0.007323\\
79	0.008967\\
80	0.008519\\
81	0.00834499999999999\\
82	0.00794\\
83	0.00803799999999999\\
84	0.00803799999999999\\
85	0.00811599999999999\\
86	0.008141\\
87	0.008537\\
88	0.008396\\
89	0.008\\
90	0.016392\\
91	0.016378\\
92	0.014884\\
93	0.018742\\
94	0.017632\\
95	0.019512\\
96	0.013316\\
97	0.019697\\
98	0.019691\\
99	0.01998\\
100	0.023274\\
101	0.03151\\
102	0.032712\\
103	0.026846\\
104	0.029592\\
105	0.02958\\
106	0.029746\\
107	0.028198\\
108	0.023854\\
109	0.023605\\
110	0.025221\\
111	0.025677\\
112	0.025667\\
113	0.024448\\
114	0.021244\\
115	0.021216\\
116	0.021844\\
117	0.021826\\
118	0.02111\\
119	0.02111\\
120	0.020978\\
121	0.020366\\
122	0.018575\\
123	0.018318\\
124	0.024911\\
125	0.029695\\
126	0.029647\\
127	0.03119\\
128	0.035548\\
129	0.037556\\
130	0.034412\\
131	0.036292\\
132	0.039021\\
133	0.039007\\
134	0.040621\\
135	0.038715\\
136	0.040309\\
137	0.037416\\
138	0.042665\\
139	0.041031\\
140	0.041013\\
141	0.041164\\
142	0.041066\\
143	0.041292\\
144	0.041624\\
145	0.04142\\
146	0.041424\\
147	0.041424\\
148	0.041368\\
149	0.041752\\
150	0.041858\\
151	0.034974\\
152	0.033466\\
153	0.03144\\
154	0.031438\\
155	0.031292\\
156	0.02915\\
157	0.029204\\
158	0.029533\\
159	0.029058\\
160	0.034056\\
161	0.03405\\
162	0.031358\\
163	0.028248\\
164	0.029734\\
165	0.025299\\
166	0.024213\\
167	0.030145\\
168	0.030103\\
169	0.029107\\
170	0.031129\\
171	0.023855\\
172	0.021562\\
173	0.020528\\
174	0.019634\\
175	0.019628\\
176	0.019364\\
177	0.019324\\
178	0.019284\\
179	0.020961\\
180	0.023712\\
181	0.025424\\
182	0.025398\\
183	0.02551\\
184	0.02096\\
185	0.013054\\
186	0.013332\\
187	0.013172\\
188	0.013425\\
189	0.013419\\
190	0.013416\\
191	0.01341\\
192	0.013525\\
193	0.013517\\
194	0.014049\\
195	0.00434499999999996\\
196	0.00429899999999996\\
197	0.00442099999999996\\
198	0.00558899999999996\\
199	0.00381799999999996\\
200	0.00561399999999996\\
201	0.00875099999999996\\
202	0.010441\\
203	0.010413\\
204	0.00888499999999996\\
205	0.015873\\
206	0.018469\\
207	0.023604\\
208	0.0253\\
209	0.023675\\
210	0.023651\\
211	0.024531\\
212	0.029026\\
213	0.028724\\
214	0.026831\\
215	0.023655\\
216	0.027527\\
217	0.027487\\
218	0.027321\\
219	0.028227\\
220	0.028369\\
221	0.029713\\
222	0.031137\\
223	0.030755\\
224	0.030669\\
225	0.030681\\
226	0.030215\\
227	0.030202\\
228	0.030282\\
229	0.030084\\
230	0.029942\\
231	0.029914\\
232	0.029357\\
233	0.031811\\
234	0.032535\\
235	0.017504\\
236	0.015957\\
237	0.024543\\
238	0.024491\\
239	0.021438\\
240	0.020726\\
241	0.020764\\
242	0.021752\\
243	0.022054\\
244	0.021412\\
245	0.021394\\
246	0.020648\\
247	0.024766\\
248	0.024802\\
249	0.024898\\
250	0.027474\\
251	0.027212\\
252	0.027174\\
253	0.027983\\
254	0.027793\\
255	0.027231\\
256	0.026907\\
257	0.025933\\
258	0.032019\\
259	0.031975\\
260	0.033174\\
261	0.03086\\
262	0.036316\\
263	0.035539\\
264	0.040082\\
265	0.036825\\
266	0.036811\\
267	0.038315\\
268	0.041659\\
269	0.039982\\
270	0.035728\\
271	0.035342\\
272	0.025753\\
273	0.025753\\
274	0.025573\\
275	0.024133\\
276	0.021579\\
277	0.015615\\
278	0.015658\\
279	0.016994\\
280	0.016984\\
281	0.014592\\
282	0.00696399999999997\\
283	0.00574299999999997\\
284	0.010291\\
285	0.013321\\
286	0.00871799999999997\\
287	0.00868999999999997\\
288	0.00735399999999997\\
289	0.00765199999999997\\
290	0.015866\\
291	0.011266\\
292	0.01528\\
293	0.00936199999999996\\
294	0.00933599999999996\\
295	0.00957799999999996\\
296	0.00553799999999997\\
297	0.00357799999999997\\
298	0.010864\\
299	0.00965299999999996\\
300	-0.00285400000000004\\
301	-0.00285800000000004\\
302	-0.00294200000000004\\
303	-0.00326400000000004\\
304	-0.00327800000000004\\
305	-0.00331600000000004\\
306	-0.00323600000000004\\
307	-0.00333200000000004\\
308	-0.00333600000000004\\
309	-0.00336400000000004\\
310	-0.00333600000000004\\
311	-0.00291500000000004\\
312	0.00143199999999996\\
313	0.000939999999999961\\
314	0.00179799999999996\\
315	0.00178999999999996\\
316	0.00207799999999996\\
317	0.00264399999999996\\
318	0.000303999999999962\\
319	-3.50000000000374e-05\\
320	-0.000280000000000038\\
321	0.00325399999999996\\
322	0.00323799999999996\\
323	0.00324999999999996\\
324	0.00255999999999996\\
325	0.00221999999999996\\
326	0.00262399999999996\\
327	0.00430799999999996\\
328	0.00430999999999996\\
329	0.00430599999999996\\
330	0.00398599999999996\\
331	0.00999599999999996\\
332	0.01066\\
333	0.00633899999999996\\
334	0.017159\\
335	0.019574\\
336	0.019564\\
337	0.017574\\
338	0.00877399999999998\\
339	0.00531199999999997\\
340	0.00501999999999997\\
341	0.00565599999999997\\
342	0.00738799999999997\\
343	0.00737999999999997\\
344	0.00746199999999997\\
345	0.00399199999999997\\
346	0.00560199999999997\\
347	0.00784699999999997\\
348	0.00908199999999997\\
349	0.00953299999999997\\
350	0.00949299999999997\\
351	0.00942099999999997\\
352	0.00986299999999997\\
353	0.010025\\
354	0.00961299999999997\\
355	0.00944399999999997\\
356	0.00945999999999997\\
357	0.00945599999999997\\
358	0.00937999999999997\\
359	0.011096\\
360	0.00339199999999998\\
361	0.00584799999999998\\
362	0.00479999999999998\\
363	0.013409\\
364	0.013387\\
365	0.012206\\
366	0.009614\\
367	0.009516\\
368	0.010546\\
369	0.012504\\
370	0.00827\\
371	0.008182\\
372	0.008456\\
373	0.011116\\
374	0.010958\\
375	0.012052\\
376	0.012918\\
377	0.01003\\
378	0.010008\\
379	0.00972\\
380	0.013582\\
381	0.01121\\
382	0.011352\\
383	0.010648\\
384	0.008068\\
385	0.008064\\
386	0.007234\\
387	0.007392\\
388	0.001586\\
389	0.001991\\
390	0.001521\\
391	0.00314499999999999\\
392	0.00313899999999999\\
393	0.00316699999999999\\
394	0.010423\\
395	0.010587\\
396	0.013541\\
397	0.010584\\
398	0.010548\\
399	0.01054\\
400	0.010558\\
401	0.010665\\
402	0.012697\\
403	0.011135\\
404	0.011349\\
405	0.011171\\
406	0.011163\\
407	0.011139\\
408	0.010787\\
409	0.010095\\
410	0.009041\\
411	0.00728\\
412	0.003838\\
413	0.003806\\
414	0.000313000000000003\\
415	0.000181000000000002\\
416	0.008057\\
417	0.00438500000000001\\
418	0.00636500000000001\\
419	0.007917\\
420	0.007881\\
421	0.010181\\
422	0.004865\\
423	-0.004053\\
424	-0.003991\\
425	-0.004073\\
426	-0.004275\\
427	-0.004285\\
428	-0.004293\\
429	-0.004215\\
430	-0.004334\\
431	-0.004342\\
432	-0.003912\\
433	-0.00398\\
434	-0.00399\\
435	-0.004364\\
436	-0.004184\\
437	-0.007044\\
438	-0.008584\\
439	-0.014232\\
440	-0.010676\\
441	-0.010688\\
442	-0.010432\\
443	-0.00852399999999999\\
444	-0.010424\\
445	-0.01066\\
446	-0.00958399999999999\\
447	-0.00989199999999999\\
448	-0.00989599999999999\\
449	-0.00994299999999999\\
450	-0.00957799999999999\\
451	-0.00928199999999999\\
452	-0.010872\\
453	-0.011298\\
454	-0.017386\\
455	-0.017386\\
456	-0.01712\\
457	-0.017117\\
458	-0.013191\\
459	-0.013288\\
460	-0.012166\\
461	-0.012561\\
462	-0.012587\\
463	-0.010322\\
464	-0.001192\\
465	0.00954999999999999\\
466	0.013186\\
467	0.00393399999999999\\
468	0.00577799999999999\\
469	0.00577199999999999\\
470	0.00844899999999999\\
471	0.00760399999999999\\
472	0.00887999999999999\\
473	0.00758399999999999\\
474	0.010392\\
475	0.012827\\
476	0.012823\\
477	0.011602\\
478	0.011366\\
479	0.017351\\
480	0.019065\\
481	0.02388\\
482	0.027029\\
483	0.027029\\
484	0.028374\\
485	0.026284\\
486	0.02997\\
487	0.02733\\
488	0.038944\\
489	0.042263\\
490	0.042261\\
491	0.043935\\
492	0.036073\\
493	0.039759\\
494	0.038369\\
495	0.042653\\
496	0.037707\\
497	0.037701\\
498	0.037547\\
499	0.038478\\
500	0.037818\\
501	0.036693\\
502	0.034393\\
503	0.040573\\
504	0.040573\\
505	0.040353\\
506	0.040603\\
507	0.031503\\
508	0.036603\\
509	0.038343\\
510	0.0308849999999999\\
511	0.0308849999999999\\
512	0.0305629999999999\\
513	0.0362149999999999\\
514	0.0406949999999999\\
515	0.0422529999999999\\
516	0.0480249999999999\\
517	0.0505239999999999\\
518	0.0505139999999999\\
519	0.0519399999999999\\
520	0.0504779999999999\\
521	0.0511939999999999\\
522	0.0495429999999999\\
523	0.048387\\
524	0.0485379999999999\\
525	0.048534\\
526	0.0482599999999999\\
527	0.050128\\
528	0.049344\\
529	0.0490919999999999\\
530	0.0485239999999999\\
531	0.0487919999999999\\
532	0.0487819999999999\\
533	0.0487819999999999\\
534	0.0489309999999999\\
535	0.0490149999999999\\
536	0.0489709999999999\\
537	0.0487689999999999\\
538	0.0489599999999999\\
539	0.0489539999999999\\
540	0.0487459999999999\\
541	0.0472939999999999\\
542	0.0473429999999999\\
543	0.0461049999999999\\
544	0.0463569999999999\\
545	0.0475699999999999\\
546	0.0475699999999999\\
547	0.0475139999999999\\
548	0.0467339999999999\\
549	0.0469389999999999\\
550	0.0458089999999999\\
551	0.0466839999999999\\
552	0.0462409999999999\\
553	0.0462289999999999\\
554	0.0460879999999999\\
555	0.054944\\
556	0.050383\\
557	0.039476\\
558	0.038823\\
559	0.039395\\
560	0.039395\\
561	0.039465\\
562	0.038773\\
563	0.038099\\
564	0.039251\\
565	0.039473\\
566	0.040049\\
567	0.040047\\
568	0.040173\\
569	0.040443\\
570	0.041253\\
571	0.041323\\
572	0.042133\\
573	0.043629\\
574	0.043625\\
575	0.044024\\
576	0.042618\\
577	0.04066\\
578	0.042951\\
579	0.039021\\
580	0.038443\\
581	0.038435\\
582	0.038471\\
583	0.037533\\
584	0.038512\\
585	0.039326\\
586	0.036826\\
587	0.038854\\
588	0.038834\\
589	0.0383239999999999\\
590	0.0348189999999999\\
591	0.0367569999999999\\
592	0.0345229999999999\\
593	0.0371239999999999\\
594	0.0372269999999999\\
595	0.0372269999999999\\
596	0.0373299999999999\\
597	0.0331399999999999\\
598	0.0326589999999999\\
599	0.0318239999999999\\
600	0.0315979999999999\\
601	0.0320009999999999\\
602	0.0320009999999999\\
603	0.0324049999999999\\
604	0.0320149999999999\\
605	0.0286849999999999\\
606	0.0270869999999999\\
607	0.0323369999999999\\
608	0.035989\\
609	0.0359729999999999\\
610	0.0346199999999999\\
611	0.034915\\
612	0.034143\\
613	0.033376\\
614	0.032092\\
615	0.033144\\
616	0.03314\\
617	0.033393\\
618	0.032781\\
619	0.032822\\
620	0.034132\\
621	0.038878\\
622	0.037719\\
623	0.037713\\
624	0.03829\\
625	0.038316\\
626	0.037694\\
627	0.037197\\
628	0.035619\\
629	0.037038\\
630	0.03702\\
631	0.037067\\
632	0.031153\\
633	0.031104\\
634	0.031004\\
635	0.034144\\
636	0.038444\\
637	0.038424\\
638	0.038424\\
639	0.039844\\
640	0.046098\\
641	0.04527\\
642	0.041826\\
643	0.041818\\
644	0.041816\\
645	0.041856\\
646	0.041491\\
647	0.041061\\
648	0.041069\\
649	0.042651\\
650	0.044559\\
651	0.044453\\
652	0.043403\\
653	0.038489\\
654	0.038485\\
655	0.041509\\
656	0.047441\\
657	0.051544\\
658	0.05151\\
659	0.050844\\
660	0.051342\\
661	0.052445\\
662	0.052329\\
663	0.052431\\
664	0.051585\\
665	0.051579\\
666	0.051663\\
667	0.050889\\
668	0.051901\\
669	0.051799\\
670	0.052051\\
671	0.052556\\
672	0.052546\\
673	0.052688\\
674	0.051886\\
675	0.051012\\
676	0.054161\\
677	0.055031\\
678	0.050242\\
679	0.050224\\
680	0.050012\\
681	0.048258\\
682	0.046716\\
683	0.037404\\
684	0.037198\\
685	0.038384\\
686	0.038378\\
687	0.043217\\
688	0.042019\\
689	0.045455\\
690	0.038395\\
691	0.044379\\
692	0.057235\\
693	0.057227\\
694	0.058118\\
695	0.0575\\
696	0.05441\\
697	0.051211\\
698	0.050015\\
699	0.047103\\
700	0.047103\\
701	0.04689\\
702	0.048862\\
703	0.052071\\
704	0.050869\\
705	0.050441\\
706	0.055192\\
707	0.055174\\
708	0.055074\\
709	0.056983\\
710	0.045562\\
711	0.046176\\
712	0.04659\\
713	0.04705\\
714	0.047048\\
715	0.046868\\
716	0.04691\\
717	0.048294\\
718	0.047738\\
719	0.045612\\
720	0.045996\\
721	0.045992\\
722	0.046052\\
723	0.045346\\
724	0.040754\\
725	0.022064\\
726	0.02155\\
727	0.033222\\
728	0.033182\\
729	0.032896\\
730	0.0246039999999999\\
731	0.0252609999999999\\
732	0.0299469999999999\\
733	0.0221889999999999\\
734	0.0121149999999999\\
735	0.0121029999999999\\
736	0.0121539999999999\\
737	0.0123569999999999\\
738	0.0121199999999999\\
739	0.0128799999999999\\
740	0.0130999999999999\\
741	0.0130579999999999\\
742	0.0130519999999999\\
743	0.0130759999999999\\
744	0.0132769999999999\\
745	0.0137009999999999\\
746	0.0127119999999999\\
747	0.0125159999999999\\
748	0.0120099999999999\\
749	0.0120079999999999\\
750	0.0108009999999999\\
751	0.0107419999999999\\
752	0.0106579999999999\\
753	0.0105929999999999\\
754	0.0104789999999999\\
755	0.0105269999999999\\
756	0.0105209999999999\\
757	0.0105449999999999\\
758	0.0106859999999999\\
759	0.0100619999999999\\
760	0.0100089999999999\\
761	0.00580799999999993\\
762	0.00386399999999993\\
763	0.00385399999999993\\
764	0.00323199999999993\\
765	0.00451599999999993\\
766	0.00539399999999993\\
767	0.00290499999999993\\
768	0.00543199999999993\\
769	0.00572999999999993\\
770	0.00572999999999993\\
771	0.00522799999999993\\
772	0.00560799999999993\\
773	0.00473799999999993\\
774	0.00523199999999993\\
775	0.00469199999999993\\
776	0.00724099999999993\\
777	0.00722499999999993\\
778	0.00746099999999993\\
779	0.00726999999999993\\
780	0.00687199999999993\\
781	0.00694199999999993\\
782	0.00513799999999993\\
783	0.00129199999999993\\
784	0.00128999999999993\\
785	0.00070699999999993\\
786	0.0109739999999999\\
787	0.00543199999999995\\
788	0.00615599999999995\\
789	0.00618399999999995\\
790	0.00582999999999995\\
791	0.00582799999999995\\
792	0.00488599999999994\\
793	0.00861999999999994\\
794	0.0100979999999999\\
795	0.00328199999999994\\
796	0.00323199999999994\\
797	0.00352599999999994\\
798	0.00347999999999994\\
799	0.00345799999999994\\
800	0.00405799999999994\\
801	0.00576199999999993\\
802	0.0105699999999999\\
803	0.00856999999999994\\
804	0.00610199999999994\\
805	0.00609999999999994\\
806	0.00560799999999994\\
807	0.00848999999999994\\
808	0.00792599999999994\\
809	0.00708699999999994\\
810	0.00703899999999994\\
811	0.00782999999999994\\
812	0.00782599999999994\\
813	0.00793699999999994\\
814	0.00975299999999994\\
815	0.00896699999999994\\
816	0.00976299999999994\\
817	0.00859399999999994\\
818	0.00768199999999994\\
819	0.00767399999999994\\
820	0.00753599999999994\\
821	0.00883599999999994\\
822	0.00478999999999994\\
823	0.00450799999999994\\
824	0.00409599999999994\\
825	0.00434599999999994\\
826	0.00433999999999994\\
827	0.00590499999999994\\
828	0.00614099999999994\\
829	0.011041\\
830	0.0130489999999999\\
831	0.00864199999999994\\
832	0.00820699999999994\\
833	0.00820499999999994\\
834	0.00864899999999994\\
835	0.00821499999999994\\
836	0.00843799999999994\\
837	0.00830599999999994\\
838	0.00682699999999994\\
839	0.00710499999999994\\
840	0.00710299999999994\\
841	0.00606699999999994\\
842	0.00664699999999994\\
843	0.00730999999999994\\
844	0.0107749999999999\\
845	0.0121009999999999\\
846	0.0150619999999999\\
847	0.0150519999999999\\
848	0.00462999999999993\\
849	0.00503599999999993\\
850	0.00747199999999993\\
851	0.00516599999999993\\
852	0.00678199999999993\\
853	0.00580999999999993\\
854	0.00578999999999993\\
855	0.00427799999999993\\
856	0.00597399999999993\\
857	0.00289599999999993\\
858	0.00621999999999993\\
859	0.00789399999999993\\
860	0.00805999999999993\\
861	0.00805599999999993\\
862	0.00800199999999993\\
863	0.00836999999999993\\
864	0.00801699999999993\\
865	0.00655499999999993\\
866	0.00648299999999993\\
867	0.00602599999999993\\
868	0.00599599999999993\\
869	0.00599299999999993\\
870	0.00586099999999993\\
871	0.00534499999999993\\
872	0.00411099999999993\\
873	0.00518999999999993\\
874	0.00574999999999993\\
875	0.00574399999999993\\
876	0.00591399999999993\\
877	0.00867599999999993\\
878	0.00944099999999993\\
879	0.00928899999999993\\
880	0.00638499999999993\\
881	0.00541299999999993\\
882	0.00540699999999993\\
883	0.00667899999999993\\
884	0.00443099999999992\\
885	0.00141099999999992\\
886	0.00272499999999992\\
887	0.00478399999999992\\
888	-0.00120200000000008\\
889	-0.00122200000000008\\
890	-0.00139200000000008\\
891	-0.00104800000000008\\
892	-0.00101600000000008\\
893	-0.00063000000000008\\
894	-0.00067800000000008\\
895	-0.00014000000000008\\
896	-0.00014000000000008\\
897	-5.20000000000797e-05\\
898	-0.00016200000000008\\
899	-0.00015000000000008\\
900	-2.00000000000797e-05\\
901	0.00018699999999992\\
902	0.00056699999999992\\
903	0.00055899999999992\\
904	0.000502999999999921\\
905	0.00119899999999992\\
906	0.00113399999999992\\
907	0.00113399999999992\\
908	0.000648999999999921\\
909	0.000354999999999921\\
910	0.000348999999999921\\
911	0.000577999999999921\\
912	0.00136999999999992\\
913	0.00356399999999991\\
914	0.00811699999999991\\
915	0.00589899999999991\\
916	0.00822799999999991\\
917	0.00822799999999991\\
918	0.00813399999999991\\
919	0.00916799999999991\\
920	0.0144289999999999\\
921	0.0187369999999999\\
922	0.0145999999999999\\
923	0.0177019999999999\\
924	0.0176999999999999\\
925	0.0177379999999999\\
926	0.0192079999999999\\
927	0.0200819999999999\\
928	0.0168439999999999\\
929	0.0147599999999999\\
930	0.00611999999999993\\
931	0.00611399999999993\\
932	0.00753099999999993\\
933	0.00287399999999993\\
934	0.00371699999999993\\
935	0.00317599999999993\\
936	0.00077999999999993\\
937	-0.000447000000000071\\
938	-0.000465000000000071\\
939	-0.00174100000000007\\
940	0.00176699999999993\\
941	0.00597899999999993\\
942	0.000835999999999924\\
943	0.00779199999999992\\
944	0.00703599999999992\\
945	0.00702799999999992\\
946	0.00724399999999992\\
947	0.0118539999999999\\
948	0.0265579999999999\\
949	0.0335689999999999\\
950	0.0371889999999999\\
951	0.0385319999999999\\
952	0.0385259999999999\\
953	0.0390769999999999\\
954	0.0386949999999999\\
955	0.0387179999999999\\
956	0.0378479999999999\\
957	0.0373179999999999\\
958	0.0369889999999999\\
959	0.0369849999999999\\
960	0.0370919999999999\\
961	0.0344839999999999\\
962	0.0318299999999999\\
963	0.0370339999999999\\
964	0.0370919999999999\\
965	0.0375529999999999\\
966	0.0375529999999999\\
967	0.0373109999999999\\
968	0.0375969999999999\\
969	0.0373789999999999\\
970	0.0379899999999999\\
971	0.038326\\
972	0.038961\\
973	0.038955\\
974	0.039057\\
975	0.038897\\
976	0.039701\\
977	0.038974\\
978	0.038038\\
979	0.0367629999999999\\
980	0.0367609999999999\\
981	0.0378499999999999\\
982	0.044071\\
983	0.027031\\
984	0.033315\\
985	0.027263\\
986	0.020911\\
987	0.020907\\
988	0.020487\\
989	0.02046\\
990	0.020412\\
991	0.020119\\
992	0.020345\\
993	0.021985\\
994	0.021983\\
995	0.020891\\
996	0.020636\\
997	0.019726\\
998	0.017695\\
999	0.018703\\
1000	0.019261\\
1001	0.019261\\
1002	0.019217\\
1003	0.018721\\
1004	0.019057\\
1005	0.019101\\
1006	0.018578\\
1007	0.016928\\
1008	0.016908\\
1009	0.017908\\
1010	0.014944\\
1011	0.015788\\
1012	0.012278\\
1013	0.011485\\
1014	0.010481\\
1015	0.010463\\
1016	0.00966699999999998\\
1017	0.00553099999999999\\
1018	0.00568699999999999\\
1019	0.00559099999999999\\
1020	0.00586499999999999\\
1021	0.00605999999999999\\
1022	0.00605799999999999\\
1023	0.00615399999999999\\
1024	0.00696599999999999\\
1025	0.00767999999999999\\
1026	0.016723\\
1027	0.015361\\
1028	0.024599\\
1029	0.024563\\
1030	0.022615\\
1031	0.01971\\
1032	0.022573\\
1033	0.027031\\
1034	0.030227\\
1035	0.024287\\
1036	0.024285\\
1037	0.023331\\
1038	0.019114\\
1039	0.00941399999999999\\
1040	0.018293\\
1041	0.022407\\
1042	0.018904\\
1043	0.018902\\
1044	0.018392\\
1045	0.020742\\
1046	0.013918\\
1047	0.015288\\
1048	0.015844\\
1049	0.016185\\
1050	0.016167\\
1051	0.014787\\
1052	0.014093\\
1053	0.014866\\
1054	0.016098\\
1055	0.015306\\
1056	0.014898\\
1057	0.014896\\
1058	0.014875\\
1059	0.014841\\
1060	0.017433\\
1061	0.029245\\
1062	0.028013\\
1063	0.026405\\
1064	0.026401\\
1065	0.025924\\
1066	0.02385\\
1067	0.02552\\
1068	0.023878\\
1069	0.024332\\
1070	0.029494\\
1071	0.029492\\
1072	0.029002\\
1073	0.028107\\
1074	0.028871\\
1075	0.029563\\
1076	0.038288\\
1077	0.0384\\
1078	0.038388\\
1079	0.038224\\
1080	0.040669\\
1081	0.040807\\
1082	0.038201\\
1083	0.037769\\
1084	0.037499\\
1085	0.037497\\
1086	0.037552\\
1087	0.03719\\
1088	0.037211\\
1089	0.037429\\
1090	0.037421\\
1091	0.037422\\
1092	0.037408\\
1093	0.037448\\
1094	0.037442\\
1095	0.037387\\
1096	0.036665\\
1097	0.036776\\
1098	0.036804\\
1099	0.0368\\
1100	0.03678\\
1101	0.036728\\
1102	0.036678\\
1103	0.036936\\
1104	0.036618\\
1105	0.0357\\
1106	0.035694\\
1107	0.035686\\
1108	0.034713\\
1109	0.035626\\
1110	0.033584\\
1111	0.038345\\
1112	0.041935\\
1113	0.041913\\
1114	0.043907\\
1115	0.042873\\
1116	0.041098\\
1117	0.042468\\
1118	0.045264\\
1119	0.053396\\
1120	0.05333\\
1121	0.053152\\
1122	0.050634\\
1123	0.053948\\
1124	0.053686\\
1125	0.056886\\
1126	0.056598\\
1127	0.056568\\
1128	0.056468\\
1129	0.055994\\
1130	0.056614\\
1131	0.056514\\
1132	0.056534\\
1133	0.051012\\
1134	0.051012\\
1135	0.05014\\
1136	0.04774\\
1137	0.04708\\
1138	0.046896\\
1139	0.055381\\
1140	0.056343\\
1141	0.056323\\
1142	0.058367\\
1143	0.056831\\
1144	0.070079\\
1145	0.066861\\
1146	0.070242\\
1147	0.069178\\
1148	0.069142\\
1149	0.066564\\
1150	0.066077\\
1151	0.066771\\
1152	0.070663\\
1153	0.081289\\
1154	0.081689\\
1155	0.081679\\
1156	0.080525\\
1157	0.083141\\
1158	0.082614\\
1159	0.07628\\
1160	0.075391\\
1161	0.078939\\
1162	0.078875\\
1163	0.078991\\
1164	0.083052\\
1165	0.08314\\
1166	0.085144\\
1167	0.08487\\
1168	0.08505\\
1169	0.085024\\
1170	0.0851\\
1171	0.083486\\
1172	0.083008\\
1173	0.082162\\
1174	0.082153\\
1175	0.083877\\
1176	0.083833\\
1177	0.083965\\
1178	0.08424\\
1179	0.084914\\
1180	0.084298\\
1181	0.084629\\
1182	0.084643\\
1183	0.084625\\
1184	0.084571\\
1185	0.084362\\
1186	0.088416\\
1187	0.082884\\
1188	0.082833\\
1189	0.083762\\
1190	0.083702\\
1191	0.08468\\
1192	0.08433\\
1193	0.088307\\
1194	0.084703\\
1195	0.082534\\
1196	0.080091\\
1197	0.080077\\
1198	0.079563\\
1199	0.072619\\
1200	0.067945\\
1201	0.072371\\
1202	0.074599\\
1203	0.071584\\
1204	0.071572\\
1205	0.071702\\
1206	0.068695\\
1207	0.074707\\
1208	0.071447\\
1209	0.074733\\
1210	0.075547\\
1211	0.075513\\
1212	0.074863\\
1213	0.074893\\
1214	0.075208\\
1215	0.078954\\
1216	0.075101\\
1217	0.075179\\
1218	0.075177\\
1219	0.075209\\
1220	0.075233\\
1221	0.075397\\
1222	0.075427\\
1223	0.075393\\
1224	0.075591\\
1225	0.075577\\
1226	0.075539\\
1227	0.075015\\
1228	0.075991\\
1229	0.078695\\
1230	0.082744\\
1231	0.08234\\
1232	0.082294\\
1233	0.083163\\
1234	0.083477\\
};
\addlegendentry{Double Q-Learning};

\addplot [color=green,dashed]
  table[row sep=crcr]{%
1	6.90480000000001e-310\\
2	1.38096e-309\\
3	-2e-06\\
4	-0.000124\\
5	-0.000126\\
6	-3.8e-05\\
7	-4e-05\\
8	-3.6e-05\\
9	0.00073\\
10	0.000602\\
11	0.000156\\
12	0.00015\\
13	-0.000423999999999999\\
14	-0.000425999999999999\\
15	-0.000477999999999999\\
16	0.000995999999999999\\
17	-0.00763199999999999\\
18	-0.00147599999999999\\
19	0.00508800000000002\\
20	0.00335200000000002\\
21	0.00333000000000002\\
22	0.00161000000000002\\
23	0.000796000000000016\\
24	0.000355000000000016\\
25	0.00100900000000002\\
26	-0.000215999999999985\\
27	-0.00301099999999998\\
28	-0.00301499999999998\\
29	-0.00381099999999998\\
30	-0.00890699999999998\\
31	-0.00539499999999998\\
32	-0.010013\\
33	-0.017693\\
34	-0.018283\\
35	-0.018309\\
36	-0.019095\\
37	-0.034197\\
38	-0.032421\\
39	-0.036095\\
40	-0.033517\\
41	-0.033624\\
42	-0.033626\\
43	-0.033976\\
44	-0.033256\\
45	-0.028036\\
46	-0.029672\\
47	-0.02924\\
48	-0.025096\\
49	-0.025096\\
50	-0.025308\\
51	-0.025364\\
52	-0.0256\\
53	-0.029322\\
54	-0.032927\\
55	-0.025173\\
56	-0.025177\\
57	-0.024757\\
58	-0.028339\\
59	-0.022109\\
60	-0.025475\\
61	-0.025028\\
62	-0.024542\\
63	-0.024542\\
64	-0.024302\\
65	-0.027086\\
66	-0.027111\\
67	-0.026265\\
68	-0.026397\\
69	-0.026733\\
70	-0.026739\\
71	-0.024531\\
72	-0.020113\\
73	-0.010917\\
74	0.00189999999999997\\
75	0.00386199999999997\\
76	0.00170999999999996\\
77	0.00169199999999996\\
78	0.00310699999999996\\
79	0.00555899999999996\\
80	0.00519699999999996\\
81	0.00474399999999996\\
82	0.00441799999999996\\
83	0.00472999999999996\\
84	0.00472799999999996\\
85	0.00497999999999996\\
86	0.00497299999999996\\
87	0.00593899999999996\\
88	0.00574599999999996\\
89	0.00470399999999996\\
90	0.00833199999999996\\
91	0.00830799999999996\\
92	0.00616999999999996\\
93	0.012808\\
94	0.011214\\
95	0.014448\\
96	0.00571599999999996\\
97	0.012924\\
98	0.01289\\
99	0.013402\\
100	0.01804\\
101	0.029943\\
102	0.031711\\
103	0.023203\\
104	0.027781\\
105	0.027757\\
106	0.028015\\
107	0.025268\\
108	0.020511\\
109	0.020133\\
110	0.022677\\
111	0.023289\\
112	0.023281\\
113	0.020749\\
114	0.013899\\
115	0.013818\\
116	0.015475\\
117	0.015411\\
118	0.01354\\
119	0.013536\\
120	0.013198\\
121	0.011466\\
122	0.00972899999999999\\
123	0.010561\\
124	0.01752\\
125	0.023401\\
126	0.023347\\
127	0.024938\\
128	0.031277\\
129	0.033761\\
130	0.029787\\
131	0.032719\\
132	0.036412\\
133	0.036396\\
134	0.038625\\
135	0.035728\\
136	0.037835\\
137	0.03333\\
138	0.040439\\
139	0.038011\\
140	0.038007\\
141	0.03816\\
142	0.038048\\
143	0.038331\\
144	0.038735\\
145	0.038478\\
146	0.038498\\
147	0.038498\\
148	0.038416\\
149	0.038702\\
150	0.038748\\
151	0.028452\\
152	0.026538\\
153	0.023316\\
154	0.023314\\
155	0.02309\\
156	0.01964\\
157	0.019436\\
158	0.020464\\
159	0.019769\\
160	0.027808\\
161	0.027806\\
162	0.023525\\
163	0.019234\\
164	0.022379\\
165	0.016148\\
166	0.013832\\
167	0.022832\\
168	0.022784\\
169	0.021284\\
170	0.024504\\
171	0.01349\\
172	0.00892400000000002\\
173	0.00619200000000001\\
174	0.00530200000000001\\
175	0.00529400000000001\\
176	0.00508400000000001\\
177	0.00502100000000001\\
178	0.00462900000000002\\
179	0.00721000000000001\\
180	0.011717\\
181	0.014489\\
182	0.014445\\
183	0.014634\\
184	0.00745600000000002\\
185	-0.00497899999999999\\
186	-0.00485499999999999\\
187	-0.00493699999999999\\
188	-0.00468399999999999\\
189	-0.00468399999999999\\
190	-0.00468599999999999\\
191	-0.00464699999999999\\
192	-0.00464899999999999\\
193	-0.00469199999999999\\
194	-0.00461599999999999\\
195	-0.019026\\
196	-0.019072\\
197	-0.018888\\
198	-0.016863\\
199	-0.019917\\
200	-0.016819\\
201	-0.011472\\
202	-0.00888799999999998\\
203	-0.00891399999999998\\
204	-0.01125\\
205	0.000123000000000033\\
206	0.00262600000000004\\
207	0.00921700000000003\\
208	0.01085\\
209	0.00879500000000004\\
210	0.00875900000000004\\
211	0.010094\\
212	0.016026\\
213	0.015824\\
214	0.01599\\
215	0.014748\\
216	0.0188400000000001\\
217	0.0188020000000001\\
218	0.0187980000000001\\
219	0.018096\\
220	0.01845\\
221	0.0189640000000001\\
222	0.018764\\
223	0.018162\\
224	0.0180760000000001\\
225	0.018074\\
226	0.017352\\
227	0.017335\\
228	0.017553\\
229	0.017315\\
230	0.0169870000000001\\
231	0.0169350000000001\\
232	0.0154330000000001\\
233	0.017739\\
234	0.019683\\
235	0.00483700000000007\\
236	-0.00292999999999993\\
237	0.00511300000000007\\
238	0.00507500000000007\\
239	0.00179000000000007\\
240	0.000812000000000066\\
241	0.000872000000000066\\
242	0.00262400000000006\\
243	0.00297400000000006\\
244	0.00196000000000006\\
245	0.00193600000000006\\
246	0.00130900000000006\\
247	0.00877300000000007\\
248	0.00854500000000007\\
249	0.00882100000000007\\
250	0.0119990000000001\\
251	0.0114210000000001\\
252	0.0113670000000001\\
253	0.0130780000000001\\
254	0.0128460000000001\\
255	0.0125980000000001\\
256	0.0127560000000001\\
257	0.0124140000000001\\
258	0.0218020000000001\\
259	0.0217360000000001\\
260	0.0233550000000001\\
261	0.0199870000000001\\
262	0.0271430000000001\\
263	0.0259860000000001\\
264	0.0320710000000001\\
265	0.0274300000000001\\
266	0.0274040000000001\\
267	0.0295970000000001\\
268	0.0339810000000001\\
269	0.0314840000000001\\
270	0.0257840000000001\\
271	0.0258860000000001\\
272	0.011271\\
273	0.011271\\
274	0.010883\\
275	0.00800100000000004\\
276	0.00269900000000005\\
277	-0.00344299999999995\\
278	-0.00345699999999995\\
279	-0.00363099999999995\\
280	-0.00364099999999995\\
281	-0.00755399999999995\\
282	-0.019111\\
283	-0.021443\\
284	-0.012872\\
285	-0.00708599999999996\\
286	-0.013783\\
287	-0.013815\\
288	-0.016007\\
289	-0.015556\\
290	-0.00189499999999997\\
291	-0.00879099999999996\\
292	-0.00210199999999996\\
293	-0.0109509999999999\\
294	-0.0109829999999999\\
295	-0.0105189999999999\\
296	-0.0163919999999999\\
297	-0.0197459999999999\\
298	-0.00915799999999994\\
299	-0.0111669999999999\\
300	-0.030891\\
301	-0.030891\\
302	-0.031251\\
303	-0.0331069999999999\\
304	-0.0331009999999999\\
305	-0.0331009999999999\\
306	-0.0331009999999999\\
307	-0.0331009999999999\\
308	-0.0331009999999999\\
309	-0.0331009999999999\\
310	-0.0332189999999999\\
311	-0.032541\\
312	-0.0250199999999999\\
313	-0.0256919999999999\\
314	-0.0246519999999999\\
315	-0.0246579999999999\\
316	-0.0242969999999999\\
317	-0.0234269999999999\\
318	-0.0265429999999999\\
319	-0.0270599999999999\\
320	-0.0273829999999999\\
321	-0.0217679999999999\\
322	-0.0217699999999999\\
323	-0.0217679999999999\\
324	-0.0225939999999999\\
325	-0.0230459999999999\\
326	-0.0232379999999999\\
327	-0.0206659999999999\\
328	-0.0206379999999999\\
329	-0.0206379999999999\\
330	-0.0211419999999999\\
331	-0.011944\\
332	-0.011302\\
333	-0.0175609999999999\\
334	-0.00262299999999996\\
335	8.30000000000327e-05\\
336	7.50000000000327e-05\\
337	-0.00287699999999997\\
338	-0.014951\\
339	-0.018271\\
340	-0.018383\\
341	-0.018951\\
342	-0.01552\\
343	-0.015544\\
344	-0.015646\\
345	-0.020727\\
346	-0.020247\\
347	-0.015955\\
348	-0.014202\\
349	-0.01306\\
350	-0.013116\\
351	-0.013249\\
352	-0.012569\\
353	-0.012187\\
354	-0.011881\\
355	-0.012266\\
356	-0.012126\\
357	-0.012126\\
358	-0.012232\\
359	-0.00983999999999996\\
360	-0.023136\\
361	-0.020803\\
362	-0.022427\\
363	-0.00895299999999993\\
364	-0.00897299999999993\\
365	-0.0108069999999999\\
366	-0.0148709999999999\\
367	-0.0149929999999999\\
368	-0.0136879999999999\\
369	-0.0111039999999999\\
370	-0.0171239999999999\\
371	-0.0172439999999999\\
372	-0.0168499999999999\\
373	-0.0156989999999999\\
374	-0.0159159999999999\\
375	-0.0153439999999999\\
376	-0.0140409999999999\\
377	-0.0184679999999999\\
378	-0.0184879999999999\\
379	-0.0189769999999999\\
380	-0.0143089999999999\\
381	-0.0171629999999999\\
382	-0.0167589999999999\\
383	-0.0185809999999999\\
384	-0.0243889999999999\\
385	-0.0244209999999999\\
386	-0.0259249999999999\\
387	-0.0255489999999999\\
388	-0.0364549999999999\\
389	-0.0355299999999999\\
390	-0.0364299999999999\\
391	-0.0327179999999999\\
392	-0.0327239999999999\\
393	-0.0326619999999999\\
394	-0.0179579999999999\\
395	-0.0177819999999999\\
396	-0.0159239999999999\\
397	-0.0187539999999999\\
398	-0.0187419999999999\\
399	-0.0187419999999999\\
400	-0.0187639999999999\\
401	-0.0185909999999999\\
402	-0.0149789999999999\\
403	-0.0175909999999999\\
404	-0.0172229999999999\\
405	-0.0174549999999999\\
406	-0.0174549999999999\\
407	-0.0174829999999999\\
408	-0.0178469999999999\\
409	-0.0188479999999999\\
410	-0.0193679999999999\\
411	-0.0200889999999999\\
412	-0.0257409999999999\\
413	-0.0257809999999999\\
414	-0.0314589999999999\\
415	-0.0316349999999999\\
416	-0.0195969999999999\\
417	-0.0232349999999999\\
418	-0.0203569999999999\\
419	-0.0181819999999999\\
420	-0.0182199999999999\\
421	-0.0149999999999999\\
422	-0.0226519999999999\\
423	-0.0346809999999999\\
424	-0.0345289999999999\\
425	-0.0346689999999999\\
426	-0.0362599999999999\\
427	-0.0362619999999999\\
428	-0.0362139999999999\\
429	-0.0358859999999999\\
430	-0.0354419999999999\\
431	-0.0354259999999999\\
432	-0.0351449999999999\\
433	-0.0352469999999999\\
434	-0.0352609999999999\\
435	-0.0358569999999999\\
436	-0.0355869999999999\\
437	-0.0405609999999999\\
438	-0.0429389999999999\\
439	-0.0527419999999999\\
440	-0.0485909999999999\\
441	-0.0486049999999999\\
442	-0.0483129999999999\\
443	-0.0462069999999999\\
444	-0.0483509999999999\\
445	-0.0499469999999999\\
446	-0.0473069999999999\\
447	-0.0495069999999999\\
448	-0.0495069999999999\\
449	-0.0497929999999999\\
450	-0.0488909999999999\\
451	-0.0459229999999999\\
452	-0.0482129999999999\\
453	-0.0488989999999999\\
454	-0.0576529999999999\\
455	-0.0576529999999999\\
456	-0.0571669999999999\\
457	-0.0571139999999999\\
458	-0.0510839999999999\\
459	-0.0511979999999999\\
460	-0.0497179999999999\\
461	-0.0503329999999999\\
462	-0.0503649999999999\\
463	-0.0464979999999999\\
464	-0.032936\\
465	-0.022588\\
466	-0.021766\\
467	-0.032586\\
468	-0.029927\\
469	-0.029935\\
470	-0.026257\\
471	-0.027377\\
472	-0.025867\\
473	-0.027737\\
474	-0.017967\\
475	-0.00287099999999995\\
476	-0.00287899999999995\\
477	-0.00436599999999995\\
478	-0.00486399999999995\\
479	0.00657200000000005\\
480	0.010092\\
481	0.0191870000000001\\
482	0.0255700000000001\\
483	0.0255680000000001\\
484	0.0282680000000001\\
485	0.0242920000000001\\
486	0.0310040000000001\\
487	0.0251240000000001\\
488	0.0478160000000001\\
489	0.0556970000000001\\
490	0.0556930000000001\\
491	0.0596250000000001\\
492	0.042221\\
493	0.049019\\
494	0.045955\\
495	0.054649\\
496	0.043085\\
497	0.043083\\
498	0.042929\\
499	0.043949\\
500	0.043229\\
501	0.041931\\
502	0.037741\\
503	0.049958\\
504	0.049958\\
505	0.049535\\
506	0.05\\
507	0.032039\\
508	0.041525\\
509	0.044973\\
510	0.031112\\
511	0.031112\\
512	0.030491\\
513	0.041555\\
514	0.050307\\
515	0.053361\\
516	0.064805\\
517	0.067273\\
518	0.067273\\
519	0.069285\\
520	0.067093\\
521	0.068531\\
522	0.065621\\
523	0.063531\\
524	0.0638010000000001\\
525	0.0637990000000001\\
526	0.0633190000000001\\
527	0.06671\\
528	0.0653300000000001\\
529	0.0648750000000001\\
530	0.0637530000000001\\
531	0.0645630000000001\\
532	0.0645510000000001\\
533	0.0644670000000001\\
534	0.0644120000000001\\
535	0.0644960000000001\\
536	0.0645320000000001\\
537	0.0643300000000001\\
538	0.0640370000000001\\
539	0.0640350000000001\\
540	0.0633340000000001\\
541	0.0597180000000001\\
542	0.0598680000000001\\
543	0.0563920000000001\\
544	0.0574000000000001\\
545	0.0636060000000001\\
546	0.0636060000000001\\
547	0.0633260000000001\\
548	0.0595820000000001\\
549	0.0606360000000001\\
550	0.0552120000000001\\
551	0.0597540000000001\\
552	0.0579350000000001\\
553	0.0579250000000001\\
554	0.0575780000000001\\
555	0.0713250000000001\\
556	0.0664700000000001\\
557	0.0512360000000001\\
558	0.0501310000000001\\
559	0.0511500000000001\\
560	0.0511480000000001\\
561	0.0515420000000001\\
562	0.0483080000000001\\
563	0.0464360000000001\\
564	0.0507880000000001\\
565	0.0516800000000001\\
566	0.0538560000000001\\
567	0.0538520000000001\\
568	0.0543280000000001\\
569	0.0554120000000001\\
570	0.0584720000000001\\
571	0.0587560000000001\\
572	0.0618160000000001\\
573	0.0684000000000001\\
574	0.0684000000000001\\
575	0.0688710000000001\\
576	0.0659930000000001\\
577	0.0623210000000001\\
578	0.0663890000000001\\
579	0.0575280000000001\\
580	0.0561580000000001\\
581	0.0561560000000001\\
582	0.0564720000000001\\
583	0.0549100000000001\\
584	0.0568350000000001\\
585	0.0582270000000001\\
586	0.0533950000000001\\
587	0.0562630000000001\\
588	0.0562370000000001\\
589	0.0554070000000001\\
590	0.0494310000000001\\
591	0.0525850000000001\\
592	0.0487790000000001\\
593	0.0530120000000001\\
594	0.0531960000000001\\
595	0.0531960000000001\\
596	0.0533800000000001\\
597	0.0465700000000001\\
598	0.0457930000000001\\
599	0.0444490000000001\\
600	0.0441750000000001\\
601	0.0452060000000001\\
602	0.0452060000000001\\
603	0.046168\\
604	0.045245\\
605	0.0373730000000001\\
606	0.0340530000000001\\
607	0.042328\\
608	0.04953\\
609	0.049514\\
610	0.047846\\
611	0.048264\\
612	0.046918\\
613	0.0457480000000001\\
614	0.043746\\
615	0.0453580000000001\\
616	0.045356\\
617	0.0457460000000001\\
618	0.044794\\
619	0.044842\\
620	0.047628\\
621	0.057824\\
622	0.055095\\
623	0.055069\\
624	0.056428\\
625	0.05648\\
626	0.053062\\
627	0.05106\\
628	0.043952\\
629	0.0460370000000001\\
630	0.046013\\
631	0.046099\\
632	0.0350050000000001\\
633	0.0349100000000001\\
634	0.0347220000000001\\
635	0.0410140000000001\\
636	0.0490980000000001\\
637	0.0490700000000001\\
638	0.0490700000000001\\
639	0.0519220000000001\\
640	0.0638480000000001\\
641	0.0630040000000001\\
642	0.0589860000000001\\
643	0.0590870000000001\\
644	0.0590850000000001\\
645	0.0591470000000001\\
646	0.0584350000000001\\
647	0.0579170000000001\\
648	0.0580270000000001\\
649	0.0614730000000001\\
650	0.0662670000000001\\
651	0.0662090000000001\\
652	0.0651610000000001\\
653	0.0582210000000001\\
654	0.0582070000000001\\
655	0.0624450000000001\\
656	0.0709110000000001\\
657	0.0767260000000001\\
658	0.0766960000000001\\
659	0.0756590000000001\\
660	0.0775930000000001\\
661	0.0825150000000001\\
662	0.0819840000000001\\
663	0.0824320000000001\\
664	0.0783170000000001\\
665	0.0783050000000001\\
666	0.0786060000000001\\
667	0.0765680000000001\\
668	0.0801980000000001\\
669	0.0797980000000001\\
670	0.0807140000000001\\
671	0.0827010000000001\\
672	0.0827010000000001\\
673	0.0830130000000001\\
674	0.0796380000000001\\
675	0.0790820000000001\\
676	0.0842800000000001\\
677	0.0852080000000001\\
678	0.0794790000000001\\
679	0.0794510000000001\\
680	0.0791510000000001\\
681	0.0767150000000001\\
682	0.0745150000000001\\
683	0.0616050000000001\\
684	0.0612630000000001\\
685	0.0656130000000001\\
686	0.0656050000000001\\
687	0.0717560000000001\\
688	0.0704120000000001\\
689	0.0806310000000002\\
690	0.0735030000000002\\
691	0.0877230000000002\\
692	0.100851\\
693	0.100849\\
694	0.10275\\
695	0.101574\\
696	0.0955620000000002\\
697	0.0896610000000002\\
698	0.0871000000000002\\
699	0.0810720000000002\\
700	0.0810720000000002\\
701	0.0805920000000002\\
702	0.0846520000000002\\
703	0.0921960000000002\\
704	0.0897360000000002\\
705	0.0883360000000002\\
706	0.0968580000000002\\
707	0.0968340000000002\\
708	0.0966440000000002\\
709	0.100521\\
710	0.0788460000000002\\
711	0.0800220000000002\\
712	0.0807260000000002\\
713	0.0816500000000002\\
714	0.0816500000000002\\
715	0.0812900000000002\\
716	0.0813750000000002\\
717	0.0842580000000002\\
718	0.0831600000000002\\
719	0.0787550000000002\\
720	0.0794270000000002\\
721	0.0794270000000002\\
722	0.0795370000000002\\
723	0.0782310000000002\\
724	0.0693450000000002\\
725	0.0544850000000001\\
726	0.0543370000000001\\
727	0.0654790000000001\\
728	0.0654650000000001\\
729	0.0652060000000001\\
730	0.0566900000000001\\
731	0.0576230000000001\\
732	0.0643120000000001\\
733	0.0537610000000001\\
734	0.0393970000000001\\
735	0.0393410000000001\\
736	0.0396350000000001\\
737	0.0404490000000001\\
738	0.0397650000000001\\
739	0.0427080000000001\\
740	0.0438530000000001\\
741	0.0436650000000001\\
742	0.0436650000000001\\
743	0.0437060000000001\\
744	0.0446030000000001\\
745	0.0462440000000001\\
746	0.0418730000000001\\
747	0.0412890000000001\\
748	0.0389530000000001\\
749	0.0389310000000001\\
750	0.0367320000000001\\
751	0.0366400000000001\\
752	0.0365160000000001\\
753	0.0364230000000001\\
754	0.0361350000000001\\
755	0.0361890000000001\\
756	0.0361890000000001\\
757	0.0362150000000001\\
758	0.0366070000000001\\
759	0.0358710000000001\\
760	0.0357530000000001\\
761	0.0253290000000001\\
762	0.0198390000000001\\
763	0.0198270000000001\\
764	0.0192510000000001\\
765	0.0205210000000001\\
766	0.0214950000000001\\
767	0.0180530000000001\\
768	0.0215090000000001\\
769	0.0219420000000001\\
770	0.0219400000000001\\
771	0.0212230000000001\\
772	0.0217410000000001\\
773	0.0204950000000001\\
774	0.0211690000000001\\
775	0.0203420000000001\\
776	0.0261740000000001\\
777	0.0261260000000001\\
778	0.0266820000000001\\
779	0.0262420000000001\\
780	0.0253340000000001\\
781	0.0256160000000001\\
782	0.0206040000000001\\
783	0.0107720000000001\\
784	0.0107680000000001\\
785	0.0086210000000001\\
786	0.0191520000000001\\
787	0.0140400000000001\\
788	0.0147660000000001\\
789	0.0147960000000001\\
790	0.0144440000000001\\
791	0.0144440000000001\\
792	0.0134660000000001\\
793	0.0187800000000001\\
794	0.0213580000000001\\
795	0.0118520000000001\\
796	0.0118160000000001\\
797	0.0116440000000001\\
798	0.0115860000000001\\
799	0.0115600000000001\\
800	0.0115380000000001\\
801	0.0110800000000001\\
802	0.0169780000000001\\
803	0.0152340000000001\\
804	0.0124080000000001\\
805	0.0124040000000001\\
806	0.0112300000000001\\
807	0.0171660000000001\\
808	0.0151340000000001\\
809	0.0132050000000001\\
810	0.0130630000000001\\
811	0.0158160000000001\\
812	0.0158080000000001\\
813	0.0162080000000001\\
814	0.0228680000000001\\
815	0.0197850000000001\\
816	0.0216000000000001\\
817	0.0178560000000001\\
818	0.0158160000000001\\
819	0.0158120000000001\\
820	0.0154500000000001\\
821	0.0177260000000001\\
822	0.00964000000000011\\
823	0.00901000000000011\\
824	0.00819400000000011\\
825	0.00984400000000011\\
826	0.00983600000000011\\
827	0.0118150000000001\\
828	0.0120960000000001\\
829	0.0246360000000001\\
830	0.0266670000000001\\
831	0.0214710000000001\\
832	0.0205390000000001\\
833	0.0205390000000001\\
834	0.0214440000000001\\
835	0.0202920000000001\\
836	0.0210550000000001\\
837	0.0206310000000001\\
838	0.0139580000000001\\
839	0.0159560000000001\\
840	0.0159560000000001\\
841	0.0137900000000001\\
842	0.0149120000000001\\
843	0.0161020000000001\\
844	0.0229590000000001\\
845	0.0254290000000001\\
846	0.0314660000000001\\
847	0.0314500000000001\\
848	0.0178850000000001\\
849	0.0184680000000001\\
850	0.0221290000000001\\
851	0.0189000000000001\\
852	0.0214350000000001\\
853	0.0197210000000001\\
854	0.0196950000000001\\
855	0.0170050000000001\\
856	0.0202550000000001\\
857	0.0147810000000001\\
858	0.0211390000000001\\
859	0.0241130000000001\\
860	0.0244350000000001\\
861	0.0244310000000001\\
862	0.0243210000000001\\
863	0.0250610000000001\\
864	0.0243300000000001\\
865	0.0214080000000001\\
866	0.0211580000000001\\
867	0.0200770000000001\\
868	0.0200310000000001\\
869	0.0200330000000001\\
870	0.0196950000000001\\
871	0.0187150000000001\\
872	0.0153520000000001\\
873	0.0178440000000001\\
874	0.0193800000000001\\
875	0.0193520000000001\\
876	0.0198230000000001\\
877	0.0263970000000001\\
878	0.0274190000000001\\
879	0.0271650000000001\\
880	0.0180630000000001\\
881	0.0157550000000001\\
882	0.0157330000000001\\
883	0.0183390000000001\\
884	0.0161150000000001\\
885	0.0117390000000001\\
886	0.0135190000000001\\
887	0.0163640000000001\\
888	0.00824400000000007\\
889	0.00821800000000007\\
890	0.00801600000000008\\
891	0.00798600000000007\\
892	0.00854600000000007\\
893	0.00884800000000007\\
894	0.00872000000000007\\
895	0.0105160000000001\\
896	0.0105160000000001\\
897	0.0108120000000001\\
898	0.0104600000000001\\
899	0.0105200000000001\\
900	0.0107560000000001\\
901	0.0115180000000001\\
902	0.0144560000000001\\
903	0.0144520000000001\\
904	0.0141660000000001\\
905	0.0182960000000001\\
906	0.0179720000000001\\
907	0.0180120000000001\\
908	0.0156200000000001\\
909	0.0149900000000001\\
910	0.0149840000000001\\
911	0.0155980000000001\\
912	0.0179440000000001\\
913	0.0243560000000001\\
914	0.0288600000000001\\
915	0.0261850000000001\\
916	0.0308530000000001\\
917	0.0308510000000001\\
918	0.0306750000000001\\
919	0.0325450000000001\\
920	0.042886\\
921	0.0471480000000001\\
922	0.0421480000000001\\
923	0.0483560000000001\\
924	0.0483560000000001\\
925	0.0484310000000001\\
926	0.0512130000000001\\
927	0.0528470000000001\\
928	0.0468090000000001\\
929	0.0429840000000001\\
930	0.0257660000000001\\
931	0.0257200000000001\\
932	0.0284320000000001\\
933	0.0205390000000001\\
934	0.0221590000000001\\
935	0.0212380000000001\\
936	0.0166960000000001\\
937	0.0146130000000001\\
938	0.0145950000000001\\
939	0.0124290000000001\\
940	0.0191190000000001\\
941	0.0262490000000001\\
942	0.0164810000000001\\
943	0.0282590000000001\\
944	0.0268370000000001\\
945	0.0268310000000001\\
946	0.0271690000000001\\
947	0.0359870000000002\\
948	0.0500430000000002\\
949	0.0572970000000002\\
950	0.0608370000000002\\
951	0.0618550000000002\\
952	0.0618450000000002\\
953	0.0625650000000002\\
954	0.0621010000000002\\
955	0.0621980000000002\\
956	0.0611260000000002\\
957	0.0604880000000002\\
958	0.0596380000000002\\
959	0.0596360000000002\\
960	0.0602690000000002\\
961	0.0537010000000002\\
962	0.0465130000000002\\
963	0.0596010000000002\\
964	0.0597580000000002\\
965	0.0609140000000002\\
966	0.0609120000000002\\
967	0.0596900000000002\\
968	0.0610300000000002\\
969	0.0599300000000002\\
970	0.0627950000000002\\
971	0.0645440000000002\\
972	0.0675250000000002\\
973	0.0675250000000002\\
974	0.0679970000000002\\
975	0.0670990000000002\\
976	0.0707510000000002\\
977	0.0666520000000002\\
978	0.0636780000000002\\
979	0.0584990000000002\\
980	0.0584650000000002\\
981	0.0605200000000002\\
982	0.0718560000000002\\
983	0.0550030000000002\\
984	0.0637790000000002\\
985	0.0555430000000002\\
986	0.0470710000000002\\
987	0.0470670000000002\\
988	0.0465330000000002\\
989	0.0465040000000002\\
990	0.0462420000000002\\
991	0.0459820000000002\\
992	0.0474720000000002\\
993	0.0538780000000002\\
994	0.0538720000000002\\
995	0.0524800000000002\\
996	0.0517720000000002\\
997	0.0494620000000002\\
998	0.0442760000000002\\
999	0.0465830000000002\\
1000	0.0483330000000002\\
1001	0.0483290000000002\\
1002	0.0482030000000002\\
1003	0.0443470000000002\\
1004	0.0461810000000002\\
1005	0.0462770000000002\\
1006	0.0451860000000002\\
1007	0.0421790000000002\\
1008	0.0421510000000002\\
1009	0.0440110000000002\\
1010	0.0381610000000002\\
1011	0.0399570000000002\\
1012	0.0332390000000002\\
1013	0.0318360000000002\\
1014	0.0301420000000002\\
1015	0.0301240000000002\\
1016	0.0291830000000002\\
1017	0.0179930000000002\\
1018	0.0189390000000002\\
1019	0.0186330000000002\\
1020	0.0191400000000002\\
1021	0.0196330000000002\\
1022	0.0196330000000002\\
1023	0.0197950000000002\\
1024	0.0222350000000002\\
1025	0.0244480000000002\\
1026	0.0353630000000002\\
1027	0.0338290000000002\\
1028	0.0436780000000002\\
1029	0.0436660000000002\\
1030	0.0416640000000002\\
1031	0.0360810000000002\\
1032	0.0411360000000002\\
1033	0.0505520000000002\\
1034	0.0568660000000002\\
1035	0.0448580000000001\\
1036	0.0448320000000001\\
1037	0.0419740000000002\\
1038	0.0234380000000001\\
1039	0.00906000000000015\\
1040	0.0257450000000002\\
1041	0.0331350000000002\\
1042	0.0265660000000002\\
1043	0.0265640000000002\\
1044	0.0256160000000002\\
1045	0.0298380000000002\\
1046	0.0170320000000002\\
1047	0.0194940000000002\\
1048	0.0205480000000002\\
1049	0.0211620000000002\\
1050	0.0211380000000002\\
1051	0.0186660000000002\\
1052	0.0174860000000002\\
1053	0.0189830000000002\\
1054	0.0213050000000002\\
1055	0.0159330000000002\\
1056	0.0137850000000002\\
1057	0.0137850000000002\\
1058	0.0140130000000002\\
1059	0.0145630000000002\\
1060	0.0193590000000002\\
1061	0.0444390000000003\\
1062	0.0432160000000003\\
1063	0.0411440000000003\\
1064	0.0411420000000003\\
1065	0.0404170000000003\\
1066	0.0371950000000002\\
1067	0.0397470000000002\\
1068	0.0371970000000003\\
1069	0.0378930000000003\\
1070	0.0459270000000002\\
1071	0.0459270000000002\\
1072	0.0451690000000002\\
1073	0.0438060000000002\\
1074	0.0450060000000002\\
1075	0.0458950000000002\\
1076	0.0553560000000002\\
1077	0.0554720000000002\\
1078	0.0554420000000002\\
1079	0.0551820000000002\\
1080	0.0586190000000002\\
1081	0.0588470000000002\\
1082	0.0550850000000002\\
1083	0.0549780000000002\\
1084	0.0548020000000002\\
1085	0.0548000000000002\\
1086	0.0547030000000002\\
1087	0.0553010000000002\\
1088	0.0552580000000002\\
1089	0.0553130000000002\\
1090	0.0553030000000002\\
1091	0.0553040000000002\\
1092	0.0553000000000002\\
1093	0.0553400000000002\\
1094	0.0553890000000002\\
1095	0.0552870000000002\\
1096	0.0547190000000002\\
1097	0.0547190000000002\\
1098	0.0546830000000002\\
1099	0.0546830000000002\\
1100	0.0546950000000002\\
1101	0.0546470000000002\\
1102	0.0546790000000002\\
1103	0.0548150000000002\\
1104	0.0548820000000002\\
1105	0.0545800000000002\\
1106	0.0545780000000002\\
1107	0.0545920000000002\\
1108	0.0537880000000002\\
1109	0.0518790000000002\\
1110	0.0489950000000002\\
1111	0.0569090000000002\\
1112	0.0623770000000002\\
1113	0.0623470000000002\\
1114	0.0649470000000002\\
1115	0.0633950000000002\\
1116	0.0610230000000002\\
1117	0.0630710000000002\\
1118	0.0664830000000002\\
1119	0.0784460000000002\\
1120	0.0783440000000002\\
1121	0.0780830000000002\\
1122	0.0739270000000002\\
1123	0.0786730000000002\\
1124	0.0789370000000002\\
1125	0.0830270000000002\\
1126	0.0823610000000002\\
1127	0.0823130000000002\\
1128	0.0820790000000002\\
1129	0.0815530000000002\\
1130	0.0819690000000002\\
1131	0.0819920000000002\\
1132	0.0818840000000002\\
1133	0.0740130000000002\\
1134	0.0740130000000002\\
1135	0.0724590000000002\\
1136	0.0684570000000002\\
1137	0.0673200000000002\\
1138	0.0664940000000002\\
1139	0.0722060000000002\\
1140	0.0738260000000002\\
1141	0.0737960000000002\\
1142	0.0771450000000002\\
1143	0.0749900000000002\\
1144	0.0932720000000002\\
1145	0.0897660000000002\\
1146	0.0948330000000002\\
1147	0.0937950000000002\\
1148	0.0937610000000002\\
1149	0.0902130000000002\\
1150	0.0896230000000002\\
1151	0.0908470000000002\\
1152	0.0962150000000002\\
1153	0.114121\\
1154	0.114211\\
1155	0.114179\\
1156	0.113253\\
1157	0.114741\\
1158	0.113731\\
1159	0.111593\\
1160	0.111306\\
1161	0.112472\\
1162	0.112382\\
1163	0.112565\\
1164	0.113363\\
1165	0.110941\\
1166	0.114722\\
1167	0.114302\\
1168	0.11499\\
1169	0.114946\\
1170	0.115496\\
1171	0.113342\\
1172	0.109788\\
1173	0.108418\\
1174	0.108485\\
1175	0.111566\\
1176	0.111516\\
1177	0.11169\\
1178	0.11416\\
1179	0.11526\\
1180	0.113336\\
1181	0.112602\\
1182	0.112803\\
1183	0.112781\\
1184	0.112667\\
1185	0.113272\\
1186	0.118182\\
1187	0.11289\\
1188	0.112961\\
1189	0.114364\\
1190	0.114292\\
1191	0.115826\\
1192	0.114768\\
1193	0.120612\\
1194	0.115752\\
1195	0.110334\\
1196	0.107036\\
1197	0.10701\\
1198	0.10626\\
1199	0.0981600000000002\\
1200	0.0911570000000002\\
1201	0.0974560000000002\\
1202	0.100768\\
1203	0.0965280000000002\\
1204	0.0965120000000002\\
1205	0.0967000000000002\\
1206	0.0923160000000002\\
1207	0.100867\\
1208	0.0947830000000002\\
1209	0.0996870000000002\\
1210	0.101532\\
1211	0.101498\\
1212	0.100762\\
1213	0.100888\\
1214	0.101305\\
1215	0.109167\\
1216	0.105255\\
1217	0.105669\\
1218	0.105669\\
1219	0.105726\\
1220	0.105687\\
1221	0.105942\\
1222	0.105894\\
1223	0.105852\\
1224	0.105552\\
1225	0.105538\\
1226	0.105536\\
1227	0.105621\\
1228	0.105249\\
1229	0.109306\\
1230	0.116267\\
1231	0.115671\\
1232	0.115625\\
1233	0.11691\\
1234	0.11748\\
};
\addlegendentry{W Q-Learning};

\addplot [color=black!60!green,dashed]
  table[row sep=crcr]{%
1	6.9311e-310\\
2	1.38622e-309\\
3	-2e-05\\
4	-0.000514\\
5	7.6e-05\\
6	0.000376\\
7	0.000364\\
8	0.00057\\
9	0.005688\\
10	0.003782\\
11	0.000819999999999997\\
12	0.000865999999999997\\
13	-0.00199200000000001\\
14	-0.00202\\
15	-0.00205\\
16	-0.000782000000000006\\
17	-0.010024\\
18	0.00651999999999999\\
19	0.013604\\
20	0.011432\\
21	0.011396\\
22	0.00922399999999999\\
23	0.00753599999999999\\
24	0.00705399999999999\\
25	0.00809999999999999\\
26	0.00674799999999999\\
27	0.00237399999999999\\
28	0.00236399999999999\\
29	0.000567999999999991\\
30	-0.00871199999999999\\
31	-0.004596\\
32	-0.009838\\
33	-0.011604\\
34	-0.012196\\
35	-0.01236\\
36	-0.013148\\
37	-0.016432\\
38	-0.011688\\
39	-0.01543\\
40	-0.00722200000000001\\
41	-0.00742400000000001\\
42	-0.00742800000000001\\
43	-0.00799000000000001\\
44	-0.00545400000000001\\
45	0.00145199999999999\\
46	-0.00205400000000002\\
47	-0.00153200000000002\\
48	0.00725199999999998\\
49	0.00719799999999998\\
50	0.00678599999999998\\
51	0.00668499999999998\\
52	0.00629099999999998\\
53	0.00222699999999997\\
54	-0.00388500000000002\\
55	0.00347299999999998\\
56	0.00347099999999998\\
57	0.00425099999999998\\
58	-0.00421600000000003\\
59	0.00221999999999997\\
60	-0.00372200000000003\\
61	-0.00269500000000003\\
62	-0.00267200000000003\\
63	-0.00267600000000003\\
64	-0.00258400000000003\\
65	-0.00425600000000003\\
66	-0.00430500000000003\\
67	-0.00388600000000003\\
68	-0.00404400000000003\\
69	-0.00431800000000003\\
70	-0.00432000000000003\\
71	-0.00385800000000003\\
72	-0.00573800000000003\\
73	0.00426199999999997\\
74	0.0192089999999999\\
75	0.0213089999999999\\
76	0.0191949999999999\\
77	0.0191949999999999\\
78	0.0206949999999999\\
79	0.0210389999999999\\
80	0.0168049999999999\\
81	0.0152209999999999\\
82	0.0142209999999999\\
83	0.0158149999999999\\
84	0.0157929999999999\\
85	0.0170769999999999\\
86	0.0171599999999999\\
87	0.0192379999999999\\
88	0.0191939999999999\\
89	0.0189779999999999\\
90	0.0291379999999999\\
91	0.0291019999999999\\
92	0.0263289999999999\\
93	0.0333169999999999\\
94	0.0312559999999999\\
95	0.0346559999999999\\
96	0.0232479999999999\\
97	0.0333899999999999\\
98	0.0333519999999999\\
99	0.0339879999999999\\
100	0.0391919999999999\\
101	0.054772\\
102	0.056898\\
103	0.046114\\
104	0.050868\\
105	0.050866\\
106	0.05112\\
107	0.047635\\
108	0.043275\\
109	0.042857\\
110	0.045793\\
111	0.046509\\
112	0.046507\\
113	0.04337\\
114	0.034074\\
115	0.034175\\
116	0.035024\\
117	0.035027\\
118	0.034028\\
119	0.034004\\
120	0.033808\\
121	0.034372\\
122	0.032033\\
123	0.038985\\
124	0.047812\\
125	0.05918\\
126	0.059172\\
127	0.062008\\
128	0.0725079999999999\\
129	0.0769489999999999\\
130	0.0701489999999999\\
131	0.0745489999999999\\
132	0.0806489999999999\\
133	0.0806489999999999\\
134	0.0843489999999999\\
135	0.0800489999999999\\
136	0.0835489999999999\\
137	0.0768489999999999\\
138	0.0885489999999999\\
139	0.0849489999999999\\
140	0.0849149999999999\\
141	0.0852069999999999\\
142	0.0845689999999999\\
143	0.0858049999999999\\
144	0.0881769999999999\\
145	0.0870569999999999\\
146	0.0870829999999999\\
147	0.0870569999999999\\
148	0.0868529999999999\\
149	0.0855789999999999\\
150	0.086237\\
151	0.075873\\
152	0.075885\\
153	0.074573\\
154	0.074553\\
155	0.074487\\
156	0.075433\\
157	0.074957\\
158	0.076452\\
159	0.0757569999999999\\
160	0.0857219999999999\\
161	0.0857219999999999\\
162	0.0804249999999999\\
163	0.0802929999999999\\
164	0.081326\\
165	0.073861\\
166	0.074286\\
167	0.083194\\
168	0.083184\\
169	0.081684\\
170	0.085759\\
171	0.074719\\
172	0.075792\\
173	0.076954\\
174	0.079832\\
175	0.07982\\
176	0.080438\\
177	0.080469\\
178	0.080411\\
179	0.0843\\
180	0.095178\\
181	0.09949\\
182	0.09949\\
183	0.099784\\
184	0.082732\\
185	0.0633879999999999\\
186	0.063936\\
187	0.0639469999999999\\
188	0.064449\\
189	0.064441\\
190	0.064403\\
191	0.064629\\
192	0.065771\\
193	0.065539\\
194	0.066419\\
195	0.051864\\
196	0.051854\\
197	0.052014\\
198	0.051489\\
199	0.050121\\
200	0.049719\\
201	0.052009\\
202	0.055009\\
203	0.054999\\
204	0.052399\\
205	0.064834\\
206	0.067734\\
207	0.074834\\
208	0.076734\\
209	0.07444\\
210	0.074438\\
211	0.077507\\
212	0.085871\\
213	0.085965\\
214	0.086056\\
215	0.085954\\
216	0.09025\\
217	0.0902\\
218	0.090351\\
219	0.089635\\
220	0.090059\\
221	0.090895\\
222	0.093233\\
223	0.091707\\
224	0.091617\\
225	0.091503\\
226	0.089833\\
227	0.090239\\
228	0.090821\\
229	0.090228\\
230	0.088902\\
231	0.088896\\
232	0.082896\\
233	0.093554\\
234	0.102386\\
235	0.085893\\
236	0.073558\\
237	0.083309\\
238	0.083279\\
239	0.079725\\
240	0.077823\\
241	0.077983\\
242	0.081255\\
243	0.081643\\
244	0.079671\\
245	0.079661\\
246	0.077203\\
247	0.096618\\
248	0.095604\\
249	0.096284\\
250	0.109978\\
251	0.108275\\
252	0.108269\\
253	0.114042\\
254	0.112904\\
255	0.11346\\
256	0.109666\\
257	0.114444\\
258	0.129444\\
259	0.129426\\
260	0.132126\\
261	0.127126\\
262	0.138526\\
263	0.136826\\
264	0.146526\\
265	0.139626\\
266	0.139626\\
267	0.142926\\
268	0.149926\\
269	0.146226\\
270	0.137188\\
271	0.136504\\
272	0.121551\\
273	0.121521\\
274	0.121327\\
275	0.121717\\
276	0.119229\\
277	0.110639\\
278	0.110566\\
279	0.109808\\
280	0.109806\\
281	0.105239\\
282	0.0935369999999999\\
283	0.0941429999999999\\
284	0.0977919999999999\\
285	0.0967719999999999\\
286	0.0893539999999999\\
287	0.0892159999999999\\
288	0.0866879999999999\\
289	0.0871669999999999\\
290	0.103333\\
291	0.0961489999999999\\
292	0.104083\\
293	0.0947999999999999\\
294	0.0946679999999999\\
295	0.0952199999999999\\
296	0.0889199999999999\\
297	0.0851099999999999\\
298	0.0964699999999999\\
299	0.0968289999999999\\
300	0.0751339999999999\\
301	0.0751299999999999\\
302	0.0746659999999999\\
303	0.0723389999999999\\
304	0.0723149999999999\\
305	0.0724229999999999\\
306	0.0721269999999999\\
307	0.0723029999999999\\
308	0.0723009999999999\\
309	0.0723409999999999\\
310	0.0718369999999999\\
311	0.0728539999999999\\
312	0.0898459999999999\\
313	0.0889459999999999\\
314	0.0909059999999999\\
315	0.0909059999999999\\
316	0.0915919999999999\\
317	0.0927679999999999\\
318	0.0874759999999999\\
319	0.0867899999999999\\
320	0.0862999999999999\\
321	0.0938459999999999\\
322	0.0938119999999999\\
323	0.0938139999999999\\
324	0.0890019999999999\\
325	0.0870739999999999\\
326	0.0844879999999999\\
327	0.0870599999999999\\
328	0.0871449999999999\\
329	0.0871449999999999\\
330	0.0865649999999999\\
331	0.0957649999999999\\
332	0.0965649999999999\\
333	0.0893679999999999\\
334	0.0895799999999999\\
335	0.0925229999999999\\
336	0.0925189999999999\\
337	0.0889049999999999\\
338	0.0665389999999999\\
339	0.0634489999999999\\
340	0.0634449999999999\\
341	0.0640149999999999\\
342	0.0687179999999999\\
343	0.0686899999999999\\
344	0.0687759999999999\\
345	0.0620499999999999\\
346	0.0617299999999999\\
347	0.0727289999999999\\
348	0.0788789999999999\\
349	0.0823859999999999\\
350	0.0823739999999999\\
351	0.0820809999999999\\
352	0.0850449999999999\\
353	0.0861509999999999\\
354	0.0868629999999999\\
355	0.0855409999999999\\
356	0.0859789999999999\\
357	0.0859769999999999\\
358	0.0858249999999999\\
359	0.0898249999999999\\
360	0.0603809999999999\\
361	0.0590349999999999\\
362	0.0572509999999999\\
363	0.0723149999999999\\
364	0.0723129999999999\\
365	0.0702889999999999\\
366	0.0657389999999999\\
367	0.0655909999999999\\
368	0.0670739999999999\\
369	0.0727539999999999\\
370	0.0578379999999999\\
371	0.0576839999999999\\
372	0.0582279999999999\\
373	0.0571919999999999\\
374	0.0567109999999999\\
375	0.0606849999999999\\
376	0.0652589999999999\\
377	0.0523979999999999\\
378	0.0523419999999999\\
379	0.0510429999999999\\
380	0.0475429999999999\\
381	0.0454549999999999\\
382	0.0456449999999999\\
383	0.044059\\
384	0.0328929999999999\\
385	0.0328489999999999\\
386	0.0324349999999999\\
387	0.0321149999999999\\
388	0.0210829999999999\\
389	0.0208569999999999\\
390	0.0208169999999999\\
391	0.0201549999999999\\
392	0.0201549999999999\\
393	0.0200029999999999\\
394	0.0360029999999999\\
395	0.0361989999999999\\
396	0.0407589999999999\\
397	0.0372209999999999\\
398	0.0372429999999999\\
399	0.0372409999999999\\
400	0.0372289999999999\\
401	0.0375289999999999\\
402	0.045129\\
403	0.040929\\
404	0.041691\\
405	0.041515\\
406	0.041513\\
407	0.041465\\
408	0.040842\\
409	0.03928\\
410	0.039452\\
411	0.0376999999999999\\
412	0.0319319999999999\\
413	0.0319239999999999\\
414	0.0260429999999999\\
415	0.0258849999999999\\
416	0.0399409999999999\\
417	0.0357409999999999\\
418	0.0391509999999999\\
419	0.0414509999999999\\
420	0.0414409999999999\\
421	0.0448409999999999\\
422	0.0358389999999999\\
423	0.0355309999999999\\
424	0.0355789999999999\\
425	0.0355389999999999\\
426	0.0351369999999999\\
427	0.0351209999999999\\
428	0.0350709999999999\\
429	0.0344129999999999\\
430	0.0336359999999999\\
431	0.0335699999999999\\
432	0.0332989999999999\\
433	0.0331999999999999\\
434	0.0331599999999999\\
435	0.0331499999999999\\
436	0.0334389999999999\\
437	0.0274769999999999\\
438	0.0251249999999999\\
439	0.0133099999999999\\
440	0.0204379999999999\\
441	0.0204359999999999\\
442	0.0209359999999999\\
443	0.0245739999999999\\
444	0.0209899999999999\\
445	0.0182479999999999\\
446	0.0230459999999999\\
447	0.0193419999999999\\
448	0.0193399999999999\\
449	0.0188549999999999\\
450	0.0204929999999999\\
451	0.0256289999999999\\
452	0.0232509999999999\\
453	0.0224749999999999\\
454	0.0133269999999999\\
455	0.0132969999999999\\
456	0.0134909999999999\\
457	0.0132909999999999\\
458	0.0200909999999999\\
459	0.0199239999999999\\
460	0.0198279999999999\\
461	0.0190979999999999\\
462	0.0189659999999999\\
463	0.0234209999999999\\
464	0.0380209999999999\\
465	0.0504029999999999\\
466	0.0561569999999999\\
467	0.0435429999999999\\
468	0.0471149999999999\\
469	0.0471149999999999\\
470	0.0519749999999999\\
471	0.0506569999999999\\
472	0.0521549999999999\\
473	0.0496709999999999\\
474	0.0575489999999999\\
475	0.0805339999999999\\
476	0.0805339999999999\\
477	0.0790339999999999\\
478	0.0784419999999999\\
479	0.0917419999999999\\
480	0.0958539999999999\\
481	0.106554\\
482	0.114096\\
483	0.114096\\
484	0.117326\\
485	0.112726\\
486	0.118322\\
487	0.110622\\
488	0.128776\\
489	0.138276\\
490	0.138272\\
491	0.143072\\
492	0.122292\\
493	0.128056\\
494	0.124132\\
495	0.136606\\
496	0.119764\\
497	0.11976\\
498	0.119672\\
499	0.120518\\
500	0.120154\\
501	0.119024\\
502	0.114424\\
503	0.116736\\
504	0.116736\\
505	0.116488\\
506	0.116988\\
507	0.113188\\
508	0.123388\\
509	0.123924\\
510	0.109004\\
511	0.109\\
512	0.108783\\
513	0.120958\\
514	0.122611\\
515	0.125975\\
516	0.128189\\
517	0.130889\\
518	0.130887\\
519	0.131129\\
520	0.128911\\
521	0.130267\\
522	0.125106\\
523	0.122706\\
524	0.123163\\
525	0.123139\\
526	0.122271\\
527	0.126108\\
528	0.12365\\
529	0.123059\\
530	0.112849\\
531	0.114623\\
532	0.114605\\
533	0.114335\\
534	0.113553\\
535	0.114321\\
536	0.115253\\
537	0.112908\\
538	0.107131\\
539	0.107109\\
540	0.104891\\
541	0.101201\\
542	0.101454\\
543	0.0963259999999998\\
544	0.0975249999999998\\
545	0.106284\\
546	0.106246\\
547	0.105826\\
548	0.1015\\
549	0.103\\
550	0.0965209999999999\\
551	0.10295\\
552	0.100147\\
553	0.100143\\
554	0.0997279999999999\\
555	0.109326\\
556	0.104275\\
557	0.101322\\
558	0.101632\\
559	0.103557\\
560	0.103533\\
561	0.104173\\
562	0.0979609999999999\\
563	0.0910649999999999\\
564	0.0977949999999999\\
565	0.0998369999999999\\
566	0.102985\\
567	0.102981\\
568	0.103617\\
569	0.106129\\
570	0.110667\\
571	0.111322\\
572	0.11577\\
573	0.131474\\
574	0.131474\\
575	0.131974\\
576	0.12854\\
577	0.12414\\
578	0.127597\\
579	0.116097\\
580	0.113045\\
581	0.113029\\
582	0.112893\\
583	0.109186\\
584	0.113698\\
585	0.116962\\
586	0.105908\\
587	0.107334\\
588	0.107246\\
589	0.10677\\
590	0.104812\\
591	0.1064\\
592	0.105124\\
593	0.107271\\
594	0.107249\\
595	0.107203\\
596	0.107181\\
597	0.103633\\
598	0.103962\\
599	0.103715\\
600	0.103408\\
601	0.104063\\
602	0.104059\\
603	0.104735\\
604	0.103771\\
605	0.0978849999999998\\
606	0.0941399999999998\\
607	0.102635\\
608	0.103697\\
609	0.103697\\
610	0.102012\\
611	0.102077\\
612	0.101681\\
613	0.101174\\
614	0.100602\\
615	0.101272\\
616	0.10126\\
617	0.101413\\
618	0.101135\\
619	0.101026\\
620	0.106758\\
621	0.118096\\
622	0.114754\\
623	0.114718\\
624	0.116416\\
625	0.116462\\
626	0.113442\\
627	0.108869\\
628	0.101187\\
629	0.102794\\
630	0.102686\\
631	0.102706\\
632	0.0989479999999999\\
633	0.0987429999999999\\
634	0.0985429999999999\\
635	0.0997199999999999\\
636	0.108232\\
637	0.108066\\
638	0.108064\\
639	0.10851\\
640	0.121081\\
641	0.120188\\
642	0.115776\\
643	0.118034\\
644	0.118032\\
645	0.118704\\
646	0.113625\\
647	0.112133\\
648	0.11308\\
649	0.11869\\
650	0.124002\\
651	0.123966\\
652	0.122965\\
653	0.115615\\
654	0.115613\\
655	0.120217\\
656	0.130285\\
657	0.13661\\
658	0.136606\\
659	0.135437\\
660	0.137081\\
661	0.145407\\
662	0.144507\\
663	0.145287\\
664	0.138587\\
665	0.138525\\
666	0.139025\\
667	0.135579\\
668	0.138993\\
669	0.138298\\
670	0.139148\\
671	0.142028\\
672	0.141972\\
673	0.142851\\
674	0.136229\\
675	0.135213\\
676	0.138859\\
677	0.139847\\
678	0.133947\\
679	0.133779\\
680	0.133479\\
681	0.132863\\
682	0.130663\\
683	0.128251\\
684	0.128203\\
685	0.140269\\
686	0.140267\\
687	0.147148\\
688	0.145736\\
689	0.152494\\
690	0.145282\\
691	0.15457\\
692	0.16797\\
693	0.16797\\
694	0.17022\\
695	0.16882\\
696	0.161662\\
697	0.156563\\
698	0.153129\\
699	0.145329\\
700	0.145329\\
701	0.144636\\
702	0.150378\\
703	0.161367\\
704	0.157884\\
705	0.157646\\
706	0.166752\\
707	0.166596\\
708	0.166396\\
709	0.167274\\
710	0.145354\\
711	0.144848\\
712	0.144846\\
713	0.14476\\
714	0.144758\\
715	0.144578\\
716	0.144678\\
717	0.14507\\
718	0.14387\\
719	0.14301\\
720	0.142816\\
721	0.142814\\
722	0.142807\\
723	0.142556\\
724	0.132818\\
725	0.121994\\
726	0.118164\\
727	0.131354\\
728	0.131352\\
729	0.131058\\
730	0.120772\\
731	0.120806\\
732	0.127564\\
733	0.125222\\
734	0.110698\\
735	0.110562\\
736	0.111353\\
737	0.113764\\
738	0.111657\\
739	0.11665\\
740	0.119043\\
741	0.118689\\
742	0.118659\\
743	0.1189\\
744	0.120327\\
745	0.124245\\
746	0.116994\\
747	0.114666\\
748	0.111062\\
749	0.11102\\
750	0.110243\\
751	0.110161\\
752	0.110187\\
753	0.110073\\
754	0.109881\\
755	0.109913\\
756	0.109901\\
757	0.109913\\
758	0.110146\\
759	0.109528\\
760	0.109613\\
761	0.0984169999999999\\
762	0.0915849999999999\\
763	0.0915489999999999\\
764	0.0902949999999999\\
765	0.0931669999999999\\
766	0.0943509999999999\\
767	0.0901249999999999\\
768	0.0968879999999999\\
769	0.0974529999999999\\
770	0.0974329999999999\\
771	0.0964479999999999\\
772	0.0974479999999999\\
773	0.0957499999999999\\
774	0.0970559999999999\\
775	0.0960149999999999\\
776	0.101324\\
777	0.101284\\
778	0.10184\\
779	0.10138\\
780	0.100486\\
781	0.10159\\
782	0.0966239999999999\\
783	0.0833519999999999\\
784	0.0833199999999999\\
785	0.0822859999999999\\
786	0.0940559999999999\\
787	0.0826579999999999\\
788	0.0836239999999999\\
789	0.0836919999999999\\
790	0.0829949999999999\\
791	0.0829729999999999\\
792	0.0812929999999999\\
793	0.0866069999999999\\
794	0.0900969999999999\\
795	0.0803949999999999\\
796	0.0802589999999999\\
797	0.0819689999999999\\
798	0.0818609999999999\\
799	0.0818129999999999\\
800	0.0827969999999999\\
801	0.0868229999999999\\
802	0.0928809999999999\\
803	0.0897129999999999\\
804	0.0867209999999999\\
805	0.0866729999999999\\
806	0.0854589999999999\\
807	0.0924799999999999\\
808	0.0895079999999999\\
809	0.0875579999999999\\
810	0.0873459999999999\\
811	0.0894629999999998\\
812	0.0894489999999998\\
813	0.0896999999999998\\
814	0.0996969999999998\\
815	0.0950419999999998\\
816	0.0967319999999998\\
817	0.0912849999999999\\
818	0.0867129999999998\\
819	0.0866969999999998\\
820	0.0864909999999998\\
821	0.0889229999999998\\
822	0.0867879999999998\\
823	0.0870289999999998\\
824	0.0868009999999998\\
825	0.0916329999999998\\
826	0.0916309999999998\\
827	0.0938559999999998\\
828	0.0941389999999998\\
829	0.102505\\
830	0.104605\\
831	0.0993049999999998\\
832	0.0982229999999998\\
833	0.0982229999999998\\
834	0.0992969999999998\\
835	0.0970049999999998\\
836	0.0976749999999998\\
837	0.0970309999999998\\
838	0.0860519999999998\\
839	0.0859649999999998\\
840	0.0859589999999998\\
841	0.0814289999999998\\
842	0.0813089999999998\\
843	0.0825189999999998\\
844	0.0842329999999998\\
845	0.0866649999999998\\
846	0.0881589999999998\\
847	0.0881569999999998\\
848	0.0744569999999998\\
849	0.0744009999999998\\
850	0.0781009999999998\\
851	0.0774969999999998\\
852	0.0764589999999998\\
853	0.0747229999999998\\
854	0.0745549999999998\\
855	0.0717549999999998\\
856	0.0722109999999998\\
857	0.0665109999999998\\
858	0.0675589999999998\\
859	0.0706589999999998\\
860	0.0707029999999998\\
861	0.0706909999999998\\
862	0.0706389999999998\\
863	0.0708129999999998\\
864	0.0705309999999998\\
865	0.0696969999999998\\
866	0.0686419999999998\\
867	0.0674779999999998\\
868	0.0674559999999998\\
869	0.0677389999999998\\
870	0.0673869999999998\\
871	0.0648109999999998\\
872	0.0611459999999998\\
873	0.0645389999999998\\
874	0.0661869999999998\\
875	0.0661849999999998\\
876	0.0666779999999998\\
877	0.0755959999999998\\
878	0.0768379999999998\\
879	0.0765269999999998\\
880	0.0698109999999998\\
881	0.0679389999999998\\
882	0.0678669999999998\\
883	0.0691509999999998\\
884	0.0667509999999998\\
885	0.0657829999999998\\
886	0.0675829999999998\\
887	0.0679429999999998\\
888	0.0597429999999998\\
889	0.0595749999999998\\
890	0.0593569999999998\\
891	0.0593839999999998\\
892	0.0598999999999998\\
893	0.0603879999999998\\
894	0.0601199999999998\\
895	0.0632639999999998\\
896	0.0632459999999998\\
897	0.0637299999999998\\
898	0.0630389999999998\\
899	0.0630889999999998\\
900	0.0641489999999998\\
901	0.0653549999999998\\
902	0.0717549999999998\\
903	0.0717509999999998\\
904	0.0711509999999998\\
905	0.0797709999999998\\
906	0.0790709999999998\\
907	0.0791649999999997\\
908	0.0742649999999997\\
909	0.0735129999999997\\
910	0.0735129999999997\\
911	0.0745869999999998\\
912	0.0787869999999997\\
913	0.0897589999999997\\
914	0.0944589999999997\\
915	0.0917589999999997\\
916	0.0973409999999997\\
917	0.0973409999999997\\
918	0.0971409999999997\\
919	0.0993409999999997\\
920	0.111587\\
921	0.115987\\
922	0.110887\\
923	0.118135\\
924	0.118135\\
925	0.118229\\
926	0.121429\\
927	0.123385\\
928	0.116385\\
929	0.11308\\
930	0.0888159999999997\\
931	0.0887599999999997\\
932	0.0893719999999997\\
933	0.0854309999999997\\
934	0.0857629999999997\\
935	0.0852659999999997\\
936	0.0840179999999997\\
937	0.0829469999999997\\
938	0.0828589999999997\\
939	0.0817469999999997\\
940	0.0833789999999997\\
941	0.0868589999999997\\
942	0.0842709999999997\\
943	0.0900469999999997\\
944	0.0895989999999997\\
945	0.0895769999999997\\
946	0.0891709999999997\\
947	0.0982809999999997\\
948	0.114281\\
949	0.121128\\
950	0.124828\\
951	0.127263\\
952	0.127223\\
953	0.127959\\
954	0.123875\\
955	0.124952\\
956	0.118171\\
957	0.115921\\
958	0.111228\\
959	0.111192\\
960	0.113641\\
961	0.107135\\
962	0.0972189999999997\\
963	0.110627\\
964	0.110881\\
965	0.112049\\
966	0.112003\\
967	0.110591\\
968	0.111757\\
969	0.110505\\
970	0.112999\\
971	0.115006\\
972	0.119804\\
973	0.119802\\
974	0.120592\\
975	0.119204\\
976	0.12534\\
977	0.118903\\
978	0.112051\\
979	0.106785\\
980	0.106737\\
981	0.10736\\
982	0.114909\\
983	0.0965919999999997\\
984	0.0981259999999997\\
985	0.0898079999999997\\
986	0.0880099999999997\\
987	0.0880079999999997\\
988	0.0878739999999997\\
989	0.0878479999999997\\
990	0.0875799999999997\\
991	0.0870779999999997\\
992	0.0884919999999997\\
993	0.101294\\
994	0.101294\\
995	0.0997439999999998\\
996	0.0991769999999998\\
997	0.0965479999999998\\
998	0.0922779999999998\\
999	0.0975009999999998\\
1000	0.0993039999999998\\
1001	0.0992999999999997\\
1002	0.0991039999999997\\
1003	0.0921619999999997\\
1004	0.0938959999999997\\
1005	0.0940719999999997\\
1006	0.0938289999999997\\
1007	0.0907149999999997\\
1008	0.0905589999999997\\
1009	0.0924269999999997\\
1010	0.0907849999999997\\
1011	0.0898789999999997\\
1012	0.0828589999999997\\
1013	0.0832209999999997\\
1014	0.0829609999999997\\
1015	0.0829589999999997\\
1016	0.0819089999999997\\
1017	0.0736449999999998\\
1018	0.0778589999999998\\
1019	0.0772019999999998\\
1020	0.0791049999999998\\
1021	0.0798479999999998\\
1022	0.0798339999999998\\
1023	0.0804999999999998\\
1024	0.0843659999999998\\
1025	0.0896469999999998\\
1026	0.10103\\
1027	0.0994299999999997\\
1028	0.110846\\
1029	0.110846\\
1030	0.108846\\
1031	0.102276\\
1032	0.106549\\
1033	0.118697\\
1034	0.126897\\
1035	0.111213\\
1036	0.111201\\
1037	0.108277\\
1038	0.0767859999999998\\
1039	0.0692539999999998\\
1040	0.0739399999999998\\
1041	0.0774199999999998\\
1042	0.0754619999999998\\
1043	0.0754139999999997\\
1044	0.0750619999999997\\
1045	0.0770299999999997\\
1046	0.0732899999999997\\
1047	0.0744179999999997\\
1048	0.0746379999999997\\
1049	0.0748839999999997\\
1050	0.0747959999999997\\
1051	0.0735979999999997\\
1052	0.0738759999999997\\
1053	0.0754389999999997\\
1054	0.0751389999999997\\
1055	0.0716409999999997\\
1056	0.0701509999999997\\
1057	0.0701469999999997\\
1058	0.0702759999999997\\
1059	0.0702599999999997\\
1060	0.0756039999999997\\
1061	0.0807759999999997\\
1062	0.0794759999999997\\
1063	0.0773569999999997\\
1064	0.0773329999999997\\
1065	0.0759879999999998\\
1066	0.0721689999999997\\
1067	0.0768129999999997\\
1068	0.0737859999999997\\
1069	0.0750379999999997\\
1070	0.0844849999999997\\
1071	0.0844649999999997\\
1072	0.0835499999999997\\
1073	0.0810389999999997\\
1074	0.0824039999999997\\
1075	0.0912149999999997\\
1076	0.102642\\
1077	0.10302\\
1078	0.10302\\
1079	0.10262\\
1080	0.10892\\
1081	0.10932\\
1082	0.102482\\
1083	0.101838\\
1084	0.101416\\
1085	0.101404\\
1086	0.101183\\
1087	0.101771\\
1088	0.101606\\
1089	0.10237\\
1090	0.102994\\
1091	0.102986\\
1092	0.102972\\
1093	0.103266\\
1094	0.103662\\
1095	0.102925\\
1096	0.101213\\
1097	0.100612\\
1098	0.100828\\
1099	0.100826\\
1100	0.10073\\
1101	0.10116\\
1102	0.100836\\
1103	0.100586\\
1104	0.0998669999999997\\
1105	0.0986389999999997\\
1106	0.0986309999999997\\
1107	0.0984689999999997\\
1108	0.0970169999999997\\
1109	0.0964889999999997\\
1110	0.0931589999999997\\
1111	0.0904489999999997\\
1112	0.0963619999999997\\
1113	0.0963619999999997\\
1114	0.0989619999999997\\
1115	0.0969629999999997\\
1116	0.0926859999999997\\
1117	0.0972339999999997\\
1118	0.101976\\
1119	0.127267\\
1120	0.127075\\
1121	0.126601\\
1122	0.118141\\
1123	0.127709\\
1124	0.129951\\
1125	0.146137\\
1126	0.143246\\
1127	0.143234\\
1128	0.142296\\
1129	0.139534\\
1130	0.1387\\
1131	0.138146\\
1132	0.136328\\
1133	0.128514\\
1134	0.128484\\
1135	0.127807\\
1136	0.128337\\
1137	0.127838\\
1138	0.1276\\
1139	0.136688\\
1140	0.138468\\
1141	0.13846\\
1142	0.142121\\
1143	0.139651\\
1144	0.139729\\
1145	0.135923\\
1146	0.141898\\
1147	0.140698\\
1148	0.140688\\
1149	0.136822\\
1150	0.136705\\
1151	0.138057\\
1152	0.144065\\
1153	0.163841\\
1154	0.165055\\
1155	0.165023\\
1156	0.164539\\
1157	0.165882\\
1158	0.164537\\
1159	0.16134\\
1160	0.160214\\
1161	0.163462\\
1162	0.163276\\
1163	0.16415\\
1164	0.18147\\
1165	0.175892\\
1166	0.191525\\
1167	0.190455\\
1168	0.193359\\
1169	0.193349\\
1170	0.195525\\
1171	0.187595\\
1172	0.174301\\
1173	0.169701\\
1174	0.170148\\
1175	0.181388\\
1176	0.18138\\
1177	0.18218\\
1178	0.191727\\
1179	0.195677\\
1180	0.188193\\
1181	0.184374\\
1182	0.185067\\
1183	0.185019\\
1184	0.184731\\
1185	0.18716\\
1186	0.205952\\
1187	0.200172\\
1188	0.200793\\
1189	0.203408\\
1190	0.203392\\
1191	0.206298\\
1192	0.201004\\
1193	0.211561\\
1194	0.205167\\
1195	0.188144\\
1196	0.182343\\
1197	0.182333\\
1198	0.181015\\
1199	0.168663\\
1200	0.155781\\
1201	0.170817\\
1202	0.176829\\
1203	0.167049\\
1204	0.167044999999999\\
1205	0.167560999999999\\
1206	0.159943\\
1207	0.180431\\
1208	0.163303\\
1209	0.171023\\
1210	0.177468\\
1211	0.177462\\
1212	0.174504\\
1213	0.174882\\
1214	0.1768\\
1215	0.196376\\
1216	0.192392\\
1217	0.193832\\
1218	0.19383\\
1219	0.194284\\
1220	0.193996\\
1221	0.196034\\
1222	0.19568\\
1223	0.195342\\
1224	0.19314\\
1225	0.19312\\
1226	0.193126\\
1227	0.194071\\
1228	0.192969\\
1229	0.196944\\
1230	0.205397\\
1231	0.204809\\
1232	0.204799\\
1233	0.206073\\
1234	0.206807\\
};
\addlegendentry{W Q-Learning with W-policy};

\end{axis}
\end{tikzpicture}%

    \caption{Cumulative profit in test set after 30 training episodes. Results are averaged over 100 experiments.}\label{F:forexCumRew}
    \end{minipage}
\end{figure}

- Moving Average Covergence/Divergence indicator:
the MACD is calculated by subtracting the 26-period exponential moving average from the 12-period moving average. A 9 period exponential moving average is used as signal line.
When the MACD falls below the signal line a long position signal is produced.
Otherwise if the MACD surpasses the signal line a short position signal is produced.

- Relative Strength Index:
the period chosen for the RSI is 20 days.
When the RSI is under the value of 30 an oversold market condition occurs, so a long position signal is produced.
Similarly, when the RSI is over 70, a short position signal is generated.
When the value is between these two bounds a close position signal is produced.

- Momentum:
the momentum uses a period of 14 days. The strategy is to be always in the market. In particular if the value of the momentum is less than zero a long position signal is generated. Otherwise  a short position signal is generated.

- Channel Commodity Index:
the CCI is used with a 20 days period. The long position signal is produced when the CCI cross and surpasses the lower bound of -100. The short position is taken when the CCI falls behind the value of 100. Otherwise the close position signal is generated.

- Stochastic Oscillator:
the Stochastic oscillator uses the high prices, low prices, and close prices with a 14 days period for the \%K line and 3  for the \%D line.
If both lines are above the value of 80 and the \%K line falls behind \%D line, then a short position signal is generated.
If both lines are below 20 and the \%K line surpasses the \%D line, a long position signal is generated. In other conditions a close position signal is generated.

- Bollinger Bands:
the Bollinger Bands signal is produced with a window size of 20.
Long position signal is produced when the closing price surpasses the upper Bollinger Band.
When the closing price falls down the lower band, then a short position signal is taken.
Otherwise a close position signal is generated.

- Moving Average Cross-Over:
we used two moving average of 20 and 200 periods. A long position signal is produced when the 20 period moving average falls behind the 200 periods one. Otherwise a “short position” signal is generated.

In Figure \ref{F:forexCumRew},  we can compare the cumulative rewards of the algorithms on the test set after 30 training episodes.
Observing the ranges of the values, we can see that the Double Q-learning agent gains and loses less than the other agents, so we can deduce that it entries in the market less often.